%%%%%%%%%%%%%%%%%%%%%%%%%%%%%%%%%%%%%%%%%%%%%%%%%%%%%%%%%%%%%%%%%%%%%%
%
% Section: First Degree Equations
%
%%%%%%%%%%%%%%%%%%%%%%%%%%%%%%%%%%%%%%%%%%%%%%%%%%%%%%%%%%%%%%%%%%%%%%

\section{Solving First Degree Equations}
\label{FirstDegreeEquations}

Usually our main goal in algebra is to solve problems by solving equations.  In this book we’ll spend quite a bit of time working on techniques to do this, but you should already have some experience.  In this chapter we will review techniques of solving first-degree equations.  Speaking a bit loosely, a \textbf{first-degree equation} \index{First Degree Equation} can be defined to be one where we don’t need to deal with exponents, roots, variables in denominators, or anything like that.

Before we begin reviewing, let’s review some of the terminology we’ll need to use.  Suppose you have a $\$25$ balance on a pre-paid phone card, and usage costs $3$ cents per minute. You can write an algebraic expression for the balance on your card based on minutes used.  If we represent the minutes used by the variable $m$, your remaining balance would be $25 - 0.03m$.  You can find out what your balance would be after any number of minutes of usage just by plugging in different values for $m$. How much would you have after $200$ minutes?  Plug in $m = 200$!  And so on.

But now suppose you want to know how many minutes you can use before your balance falls to $\$10$. So, in other words, you need to solve the equation:
$$25 - 0.03m = 10$$

Now, most values that we might plug in for $m$ will not make this equation true.  If we plug in, say, $m = 200$, we would get:
\begin{align*}
	25 - 0.03(200) & = 10 \\
	25 - 6 & = 10 \\
	19 & = 10	
\end{align*}

which is nonsense. If you use $200$ minutes, we can see that your remaining balance is $\$19$, not the $\$10$ we were looking for. $200$ minutes is not the amount of usage you were looking for.

On the other hand, if we plug in $m = 500$, we get:
\begin{align*}
	25 - 0.03(500) & = 10 \\
	25 - 15 & = 10 \\
	10 & = 10
\end{align*}

which is exactly what we would have hoped for. If you use $500$ minutes, your remaining balance is $\$10$. $500$ minutes is the amount of usage you were looking for.

In algebra, we say that a given value \textbf{satisfies} \index{Equations!Satisfies} an equation if and only if the result is true when we plug that value into the equation. We can also say that a given value is a solution for the equation. So, in this example, we would say that \quotes{$500$ satisfies the equation $25 - 0.03m = 10$} or that \quotes{$500$ is a solution for this equation.}

Likewise we would say that \quotes{$200$ does not satisfy (or is not a solution for) this equation.}

The set of all solutions for a given equation is called its solution set. We would say that the solution set of $25 - 0.03m = 10$ is $\{500\}$ (this solution set has just one member.)  We know that $500$ must be the only member of this solution set because common sense tells us that there is only one amount of usage that would result in having $\$10$ left.

Most first-degree equations have only one solution, but equations can have multiple solutions. For example, suppose that using physics someone has determined that the expression $64t - 16t^2$ gives the height of a ball (in feet) $t$ seconds after it is thrown upward, and we want to know when the ball will be at a height of $48$ feet. So we would need to know the solution set for the equation $64-16t^2 = 48$. (Note that this is not a first-degree equation, because of the squared term). It turns out that this equation has two solutions, $t=1$ and $t=3$, so the ball is at a height of $48$ feet $1$ second and $3$ seconds after it is thrown. So the solution set is $\{1,3\}$. It makes perfect sense that there would be two solutions for this equation, since the ball would reach this height twice, once on the way up, and once on the way down. Common sense also tells us that these must be the only two solutions to the equation, since the ball would not be at this height at any other time.

\exam{\label{FirstDegreeEquationsExample1}Determine which, if any, of the values $1$, $2$, $3$, $4$, or $5$ satisfy the equation $3x-2=10$}

\indenttext{To do this, we will need to substitute each of the given values into the given equation. Plugging in $x=1$ we get:
\begin{align*}
	3(1)-2 & = 10 \\
	3-2 & = 10 \\
	1 & = 10
\end{align*}

which clearly is a false statement and we see that $x=1$ does not satisfy the equation.

Moving on to $x=2$, we get:
\begin{align*}
	3(2)-2 & = 10 \\
	4-2 & = 10 \\
	4 & = 10
\end{align*}
which again is a false statement and we conclude that $x=2$ does not satisfy the equation.

You can repeat this process yourself for the remaining values to try. You should see that only $x=4$ satisfies the equation; the other given values do not.
}

%%%%%%%%%%%%%%%%%%%%%%%%%%%%%%%%%%%%%%%%%%%%%%%%%%%%%%%%%%%%%%%%%%%%%%
%
% Subsection: First Degree Equations: First Degree Equations: Balance Principle
%
%%%%%%%%%%%%%%%%%%%%%%%%%%%%%%%%%%%%%%%%%%%%%%%%%%%%%%%%%%%%%%%%%%%%%%

\subsection{Solving Equations Using the Balance Principle}

Of course, it’s one thing to check whether or not a given value satisfies an equation. It’s quite another to figure out an equation’s solution(s) without being given values to try. If we had not been given the value of $m=500$ to try in our phone care example, how would we have known to try it?  This is the great challenge of algebra.

\begin{SCfigure}[0.6][H]
	\centering
	\input{Sections/FirstDegreeEquationsImages/TextPage.pdf_tex}
	%\includegraphics[scale=1]{Sections/FirstDegreeEquationsImages/TextPage.png}	
	\caption{The picture shown above is a page from \quotes{\textit{Al-Kit\={a}b al-mukhta$\stackrel[^\textrm{.}]{}{\textrm{s}}$ar f\={i} $\stackrel[^\textrm{.}]{}{\textrm{h}}$is\={a}b al-jabr wa-l-muq\={a}bala}} written in the early-mid 800’s C.E. by the Persian mathematician Ab\={u} Abdall\={a}h Mu$\stackrel[^\textrm{.}]{}{\textrm{h}}$ammad ibn M\={u}s\={a} al-Khw\={a}rizm\={i}. His work is generally considered to be the first major step toward modern algebra (the word \quotes{algebra} actually comes from the word \quotes{al-jabr} in the title.)  The book’s Arabic title translates into English as 'The Compendious Book on Calculation by Completion and Balancing'.}
\end{SCfigure}

%\footnote{By Al-Khwarizmi - Esposito, John L. , ed. (1999) The Oxford History of Islam, Oxford University Press ISBN: 0195107993. ; April 2006 (upload date) by Spm, Public Domain, https://commons.wikimedia.org/w/index.php?curid=716423}

\quotes{Balancing} is one of the most fundamental principles used in algebra.  The basic idea is this: think of an equation as a balanced scale.  The two sides of the equation are equal, and so if we placed them on an imaginary scale, the scale would balance.  If you change the value of one side of the equation (say by adding or subtracting something to it, or multiplying or dividing it by something) the scale will no longer be in balance.  But you can recover that balance by making the same change to the other side.

This all boils down to the following rule:

\begin{definition}
	\index{Equation!Balance Principle}
	\textbf{\underline{The Balance Principle}}\\
	\bigskip
	You can make any change you like to one side of an equation, as long as you make an equivalent change to the other side.
\end{definition}

In algebra, we exploit this rule all the time. Let’s walk through a very basic example to review how it works, and then we’ll proceed to some more complicated problems.

Suppose we want to solve:
$$25 - 0.03m = 10$$

and let’s also forget that anyone ever said anything about trying any particular values as solutions.  Let’s just suppose we are given the equation and are asked to determine its solution.

So, our goal is to get an equation which directly tells us the correct value of $m$. So, what we will do here is to try to transform the given equation into something that just has the variable $m$ isolated on one side, and a number on the other side.

Let’s begin by getting rid of the $25$. We can eliminate that $25$ on the left by subtracting it - and we are allowed to do that so long as we also subtract $25$ from the right, to keep the equation balanced. So:
\begin{align*}
	25-0.03m\bm{-25} & = 10\bm{-25}\\
	-0.03m & = -15
\end{align*}

Now, we don’t have $m$ isolated yet, but we’re closer. We now just have to get rid of the coefficient $-0.03$. Now, it is a common mistake to try to do this by adding $0.03$ to both sides, but that won’t work because $-0.03m$ and $0.03$ are not like terms, and so they cannot be combined.  But we can divide both sides by $-0.03$:
$$\frac{-0.03m}{\bm{-0.03}} = \frac{-15}{\bm{-0.03}}$$

Which gives us:
$$m = 500$$

So we conclude that the solution is $m = 500$.  Stating this solution in terms appropriate for the question at hand we might phrase it as \quotes{$500$ minutes of usage}. We can also feel confident that this is the only solution, because this is the only solution that we arrived at by following the methods of algebra.

Using the principle of balance to transform an equation into one which explicitly states its solution is one of the primary methods of algebra, and really the only technique we will be using in this chapter. However, with more complicated equations, the strategies to use to get the variable by itself may be more involved. The following examples will illustrate how the principle of balance can be used effectively with some of these more complicated equations.

\exam{\label{FirstDegreeEquationsExample2}Solve for $x$: $$5x-11+3(2x-3)=2$$}

\indenttext{
	Before putting the principle of balance to work on this, it would be best to simplify by distributing
	and combining like terms. So:
	\begin{align*}
		5x-11+3(2x-3)&=2\\
		5x-11+6x-9&=2\\
		11x-20&=2
	\end{align*}

	From here, we just need to add $20$ and then divide by $11$ to get isolate the $x$:
	\begin{align*}
		11x-20&=2\\
		11x&=22\\
		x&=2
	\end{align*}

	It never hurts to check your answers by substituting your solution back into the original equation:
	\begin{align*}
		5x-11+3(2x-3)&=2\\
		5(2)-11+3(2(2)-3)&=2\\
		2&=2
	\end{align*}

	Since the equation holds true when we substitute in $x=2$, the solution checks as correct.
}

Often we run into equations where the variable appears on both sides. In those cases, the best strategy is usually to first simply both sides as much as possible, and then collect all the variable terms together on one side, and all the non-variable terms on the other. The following example will illustrate:

\exam{\label{FirstDegreeEquationsExample3}Solve for $z$: $$5(z+1)-16=3z-5$$}

\indenttext{
	\begin{align*}
		5(z+1)-16&=3z-5\\
		5z+5-16&=3z-5\\
		5z-11&=3z-5
	\end{align*}

	It doesn’t matter which side we collect the $z$’s on; either is fine.  It is more common to collect them on the left, so we’ll subtract $3z$ from both sides to get:
	$$2z-11=-5$$

	Now since we’ve collected $z$’s on the left, we’ll add $11$ to both sides to get rid of the $-11$ on the left:
	$$2z=6$$

	And now dividing both sides by 2 we get:
	$$z=3$$

	A check confirms the result:
	\begin{align*}
		5(z+1)-16&=3z-5\\
		5(3+1)-16&=3(3)-5\\
		4&=4
	\end{align*}
}

Let’s try one more example before moving on.

\exam{\label{FirstDegreeEquationsExample4}Solve for $t$: $$4t-3-7(2-t)=2(7t+5)-(t-4)$$}

\indenttext{
	As before, we’ll simplify each side first. Remember that you need to be careful about the negatives when distributing!
	\begin{align*}
		4t-3-7(2-t)&=2(7t+5)-(t-4)\\
		\\
		4t-3-14+7t&=14t+10-t+4\\
		\\
		11t-17&=13t+14
	\end{align*}

	Now, as mentioned in the previous example, there is a tendency to collect the variable on the left.  So, if we collect the variable on the left we subtract $13t$ from both sides to get:
	$$-2t-17=14$$

	Then add $17$ to both sides:
	$$-2t=31$$

	And now divide by $-2$:
	$$t=\frac{31}{-2}$$ 
	
	which is more commonly rewritten as: 
	$$t=-\frac{31}{2}=\frac{-31}{2}=-15.5$$

	Some people feel a strong preference against negative coefficients if they can be avoided, and so might prefer to subtract $11t$ from both sides. That is also fine, and in fact we can do this and then switch the equation’s sides if we want to. So this is also a correct solution:
	\begin{align*}
		4t-3-7(2-t)&=2(7t+5)-(t-4)\\
		\\
		4t-3-14+7t&=14t+10-t+4\\
		\\
		11t-17&=13t+14\\
		\\
		-17&=2t+14\\
		\\
		-31&=2t\\
		\\
		-\frac{31}{2}&=t
	\end{align*}

	Finally, to check. We’ll substitute this solution into the original equation. Since the arithmetic gets tedious, typing each side’s expression into the calculator is probably the best choice; remember though to be careful to use parentheses appropriately when substituting in this negative number:
	\begin{align*}
		4t-3-7(2-t)&=2(7t+5)-(t-4)\\
		\\
		4\left(\frac{-31}{2}\right)-3-7\left[2-\left(\frac{-31}{2}\right)\right]&=2\left[7\left(\frac{-31}{2}\right)+5\right]-\left[\left(\frac{-31}{2}\right)-4\right]\\
		\\
		-187.5&=-187.5
	\end{align*}
}

As the choice discussed in the above example makes clear, we often have some choice about the order we do things. While the examples given in this book are meant to illustrate sound strategy, all of these problems can be correctly solved in other ways as well. \textbf{As long as the steps you take follow the laws of algebra} you will arrive at a correct solutions.

%%%%%%%%%%%%%%%%%%%%%%%%%%%%%%%%%%%%%%%%%%%%%%%%%%%%%%%%%%%%%%%%%%%%%%
%
% Subsection: First Degree Equations: Solving Equations Involving Fractions
%
%%%%%%%%%%%%%%%%%%%%%%%%%%%%%%%%%%%%%%%%%%%%%%%%%%%%%%%%%%%%%%%%%%%%%%

\subsection{Solving Equations Involving Fractions}

When equations involve fractions, the same principles apply, but the arithmetic can quickly become tedious. Arithmetic with fractions is much more work than arithmetic with whole numbers, even when using a calculator.

We can avoid fraction arithmetic by multiplying both sides of the equation through by a common denominator.  This technique is known as \textbf{clearing the fractions}. The following example will illustrate.

\exam{\label{FirstDegreeEquationsExample5}Solve for $y$: $$\frac{2}{3}y-\frac{7}{5}=\frac{1}{6}y+\frac{11}{4}$$ }

\indenttext{
	Among the denominators $3,5,6,$ and $4$ is 60.  So, to clear the fractions, we start by multiplying both sides by 60.
	\begin{align*}
		60\left(\frac{2}{3}y-\frac{7}{5}\right)&=60\left(\frac{1}{6}y+\frac{11}{4}\right) \\
		\\
		60\left(\frac{2}{3}y\right)-60\left(\frac{7}{5}\right)&=60\left(\frac{1}{6}y\right)+60\left(\frac{11}{4}\right)\\
		\\
		\frac{120}{3}y-\frac{420}{5}&=\frac{60}{6}y+\frac{660}{4}\\
		\\
		40y-84&=10y+165
	\end{align*}

	Now, notice what happened when we multiplied; because all of the denominators divide 60 evenly, the result in each term is a whole number.  Now we can just continue as we have before:
	\begin{align*}
		40y-84&=10y+165 \\
		\\
		30y-84&=165 \\
		\\
		30y&=249 \\
		\\
		y&=\frac{249}{30}=\frac{83}{10}=8.3
	\end{align*}

	The reduction of the fraction to lowest terms is useful for expressing the exact answer to the equation.  This answer can be checked by substituting back into the original equation (use your calculator for the arithmetic!)
}

It is not absolutely necessary to use the lowest common denominator to clear the fractions from an equation; any common denominator will work. So in the previous example, we could also have just multiplied all the denominators together to get a common denominator and so multiplied both sides by $(3)(5)(6)(4) = 360$. This would have worked out just fine, though the numbers we would have gotten would have been larger after clearing. The benefit of using the lowest common denominator is that it keeps the numbers as small as possible. That is nice, but not mandatory.

Be careful when clearing the fractions if the original equation contains parentheses. When fractions fall inside parentheses, sometimes you may need to multiply through by a common denominator more than once to fully clear all the fractions. The following two examples will illustrate this.

\exam{\label{FirstDegreeEquationsExample6}Solve for $x$: $$\frac{2}{3}x-1=\frac{3}{5}\left(\frac{1}{2}x+1\right)$$}

\indenttext{
	Here the lowest common denominator is $30$, so we multiply both sides:
	$$30\left(\frac{2}{3}x-1\right)=30\left[\frac{3}{5}\left(\frac{1}{2}x+1\right)\right]$$

	When simplifying we generally work from inside parentheses out, but here we are better off multiplying
	30 into the outermost parenthesis on the right first:
	$$30\left(\frac{2}{3}x-1\right)=18\left(\frac{1}{2}x+1\right)$$

	Now continuing as we have before:
	\begin{align*}
		20x-30&=9x+18\\
		\\
		11x-30&=18\\
		\\
		11x&=48\\
		\\
		x&=\frac{48}{11}=4.3636...
	\end{align*}
}

\exam{\label{FirstDegreeEquationsExample7}Solve for $t$: $$\frac{1}{4}\left(\frac{3}{8}t+1\right)=\frac{2}{3}t-1$$}

\indenttext{
	Here the lowest common denominator is $24$, so we multiply both sides:
	\begin{align*}
		24\left[\frac{1}{4}\left(\frac{3}{8}t+1\right)\right]&=24\left(\frac{2}{3}t-1\right)\\
		\\
		6\left(\frac{3}{8}t+1\right)&=24\left(\frac{2}{3}t-1\right)\\
		\\
		\frac{9}{4}t+6&=16t-24	
	\end{align*}

	Now the fractions turn out to no be fully cleared out.  So, we have to multiply a second time, this time by $4$ to get:
	\begin{align*}
		4\left(\frac{9}{4}t+6\right)&=4\left(16t-24\right)\\	
		\\
		9t+24&=64t-96
	\end{align*}

	Now, with the fractions cleared, we proceed:
	\begin{align*}
		9t+24&=64t-96\\
		\\
		120&=55t\\
		\\
		t&=\frac{120}{55}=\frac{24}{11}=2.1818...
	\end{align*}
}

Clearing the fractions is optional. All of the above examples can be done without it, just working through the algebra with fraction arithmetic. Most people find it easier to clear the fractions instead, but this is a matter of choice not necessity. The choice is yours.

%%%%%%%%%%%%%%%%%%%%%%%%%%%%%%%%%%%%%%%%%%%%%%%%%%%%%%%%%%%%%%%%%%%%%%
%
% Subsection: First Degree Equations: Identities and Contradictions
%
%%%%%%%%%%%%%%%%%%%%%%%%%%%%%%%%%%%%%%%%%%%%%%%%%%%%%%%%%%%%%%%%%%%%%%

\subsection{Identities and Contradictions}

Not all problems can be solved! This fact of life shows up with equations as well. Suppose you are asked to find a solution for the equation:

$$ x = x + 1 $$

Without even trying any algebra on it, we can see we have a problem. The equation says that $x$ must be equal to $x+1$. Any solution to this equation would have to be equal to itself plus one! That’s impossible – no number is one bigger than itself. So there is no way that this equation could ever have a solution. Equations like this one that have no solutions are called \textbf{inconsistent} \index{Equation!Inconsistent} or \textbf{contradictions} \index{Equation!Contradiction}.

Now, not all inconsistent equations are as transparent as this one. For example, the equation:
$$5(x-2)+3(x+1)=2x+1-3(11-2x)$$

is also inconsistent. But while it is pretty obvious that no number can equal itself plus one, it is not even remotely obvious that it is impossible for five times a two less than a number plus three times one more than that number to equal one more than twice the number less three times eleven take away twice the number! So, how can we tell if an equation is inconsistent?

Let’s see what would happen if we tried to use algebra to solve this equation:
\begin{align*}
	5(x-2)+3(x+1)&=2x+1-3(11-2x)\\
	\\
	5x-10+3x+3&=2x+1-33+6x\\
	\\
	8x-7&=8x-32\\
	\\
	-7&=-32
\end{align*}

When we try to get the $x$'s on one side by subtracting $8x$ from both sides and get an equality which is nonsense as $-7$ is not equal to $-32$.  And \underline{that} is what tells us that this equation is inconsistent. We followed the methods of algebra in good faith, assuming that there was a solution, and those methods led us to the absurd conclusion. So, if you believe this equation has a solution, you then also have to believe that $-7=-32$!  Unless you are willing to allow this as true, you have to conclude that the equation has no solution.

The solution set of an inconsistent equation is an empty set. Since there are no solutions, the set has no members. This set is commonly called the \textbf{empty set} or the \textbf{null set} and is represented with the symbol $\emptyset$. Be careful not to confuse the empty set with the number $0$. They look similar, but they are not at all the same thing. It is correct to say that this equation’s solution set is $\emptyset$, but it is completely wrong to say that the solution set is $0$ or to say that $0$ is a solution to the equation.

On the other hand, sometimes an equation has lots and lots of solutions. For example, consider the equation:
$$6x+10=2(3x+5)$$

It’s probably not immediately obvious that this is not an ordinary equation, but think about what would happen if you distributed the $2$ into the parentheses on the right. You’d wind up with the equation:
$$6x+10=6x+10$$

Now, this is true no matter what you substitute in for $x$! Any real number is a solution for it. Equations like this are called \textbf{identities} \index{Equation!Identity}. Because any real number satisfies an identity, we say that its solution set is the set of all real numbers, which we often indicate using the symbol $\mathbb{R}$.

Just as many inconsistent equations are not obviously so, many identities are not obvious identities. For example, this equation is an identity:
$$4x-2(3x-5)=7x-3(3x-3)+1$$

but that’s not obvious at all. How can we tell?

Well, suppose we set out to solve this equation:
\begin{align*}
	4x-2(3x-5)&=7x-3(3x-3)+1\\
	\\
	4x-6x+10&=7x-9x+9+1\\
	\\
	-2x+10&=-2x+10
\end{align*}

Now, at this point we have the same expression on both sides of the equation, which is a dead giveaway.  If the two sides are the same, then they will be equal irrespective of what $x$ might be. If, though, you don’t notice that they are the same, you could continue and add $2x$ to both sides and get:
$$10=10$$

No matter what $x$ is, $10$ is equal to itself. So, this also tells us that we have an identity on our hands!

(Be careful here! It is a common mistake to misread this last equation as saying that the solution is $10$. This last equation does not say that $x=10$, it says that $10=10$. That is true no matter what $x$ is, which is why we conclude that this is an identity.)

When given an equation to solve, it is always a possibility that the equation may turn out to be an inconsistent equation or an identity. But if it is, that will become apparent as you work to solve it.  If it is inconsistent, that will become apparent once the algebra leads you to an absurd equation; if it is inconsistent that will become apparent once the algebra leads you to an equation whose sides are both the same. If it is an ordinary equation with a solution (called a \textbf{conditional equation} \index{Equation!Conditional}) the algebra will take you there, too.

\exam{\label{FirstDegreeEquationsExample8}Solve for $P$: $$3P-5(4P-3)=-3(5P-1)-2P$$}

\indenttext{
	\begin{align*}
		3P-5(4P-3)&=-3(5P-1)-2P\\
		\\
		3P-20P+15&=-15P+3-2P\\
		\\
		-17P+15&=-17P+3\\
	\end{align*}

	At this point you may notice that we have an impossible situation in the equation, but it does no harm to continue. Adding $17P$ to both sides we get:
	$$15=3$$

	which is clearly impossible, no matter what value we might substitute in for $P$. The equation is a contradiction; its solution set is $\emptyset$.
}

%%%%%%%%%%%%%%%%%%%%%%%%%%%%%%%%%%%%%%%%%%%%%%%%%%%%%%%%%%%%%%%%%%%%%%
%
% Subsection: First Degree Equations: Solve for variable
%
%%%%%%%%%%%%%%%%%%%%%%%%%%%%%%%%%%%%%%%%%%%%%%%%%%%%%%%%%%%%%%%%%%%%%%

\subsection{Solving for One Variable in Terms of Others}

In Section \ref{AlgebraicSubstitution} we used the formula:
$$W=210-\frac{m}{3}$$

which expresses the relationship between $m$, minutes of vigorous exercise per day, and $W$, a given person’s projected weight after one year. In the examples of that section, we substituted in values of $m$ and found the resulting values of $W$.

This works great if you know the minutes of exercise and want to know the projected weight, but what if the situation is reversed? Suppose we have in mind a desired weight and want to know the minutes of exercise that would be needed to achieve that. For example, suppose we want to know the minutes of exercise needed to achieve a weight of $185$ pounds? Well, we could substitute in $W=185$ to get the equation:
$$185=210-\frac{m}{3}$$

which we could then solve for $m$:
\begin{align*}
	185&=210-\frac{m}{3} \\
	\\
	3(185) &= 3 \left(210 - \frac{m}{3}\right) \\
	\\
	555 &= 630 - m \\
	\\
	-75 &= -m \\
	\\
	m &= 75
\end{align*}

So, $185=210-\frac{m}{3}$ implies that $75$ minutes of daily exercise would be required.

Now, that’s fine if we only need to find the minutes required for one weight, but what if we want to be able to find the minutes required for several different weights? It would be tedious to have to work through essentially the same algebra over and over again. Instead, why not rework the original equation so that it is set up to plug in $W$ and give $m$ as a result. We call this solving for $m$ in terms of $W$.

The algebra principles and methods we use are essentially the same as we’ve been using, except that instead of getting a number as a solution, we expect the result to be a formula:
\begin{align*}
	W&=210-\frac{m}{3} \\
	\\
	3W &= 3 \left(210 - \frac{m}{3}\right) \\
	\\
	3W &= 630 - m \\
	\\
	3W-630 &= -m \\
	\\
	(-1)(3w-630)&=(-1)(-m)\\
	\\
	-3W+630&=m
\end{align*}

which we can now rewrite as:
$$m=-3W+630 \text{ or } m=630-3W$$

It should be apparent that this new equation is set up to give values of $m$ readily once a value of $W$ is plugged in.

\exam{\label{FirstDegreeEquationsExample9}The equation $C=\frac{5}{9}(F-32)$ gives $C$, the temperature in degrees Celsius, in terms of $F$, the temperature in degrees Fahrenheit. Solve this equation for $F$ in terms of $C$.}

\indenttext{
	\begin{align*}
		C&=\frac{5}{9}(F-32)\\
		\\
		9C&=9\left(\frac{5}{9}(F-32)\right)\\
		\\
		9C&=5(F-32)\\
		\\
		9C&=5F-160\\
		\\
		9C+160&=5F\\
		\\
		\frac{9C+160}{5}&=F
	\end{align*}

	which we can then rewrite as:
	$$F=\frac{9C+160}{5}$$

	This is a correct solution. It would also be fine to divide the $5$ into each term, giving a final answer of $F=\frac{9}{5}C+32$ or to rewrite the division by $5$ as multiplication by $\frac{1}{5}$, giving a final answer of $F=\frac{1}{5}\left(9C+160\right)$.
}

Most of the time in this course our equations will involve at most two variables, but there is no reason we can’t do this with equations involving more variables.

\exam{\label{FirstDegreeEquationsExample10} Solve for $L$ in terms of the other variables in $P=2L+2W$}

\indenttext{
	\begin{align*}
		P&=2L+2W\\
		\\
		P-2W&=2L\\
		\\
		\frac{P-2W}{2}&=L\\
	\end{align*}

	$$L=\frac{P-2W}{2} \text{ or } L=\frac{1}{2}\left(P-2W\right)$$
}

%%%%%%%%%%%%%%%%%%%%%%%%%%%%%%%%%%%%%%%%%%%%%%%%%%%%%%%%%%%%%%%%%%%%%%
%
% Subsection: First Degree Equations: Exercises
%
%%%%%%%%%%%%%%%%%%%%%%%%%%%%%%%%%%%%%%%%%%%%%%%%%%%%%%%%%%%%%%%%%%%%%%

\clearpage

\subsection{Exercises}

\subsubsection*{Solutions to Equations}
Determine which, if any, of the given values satisfy the given equations.

\ex{$3x-7=8$; \indent $x=1,2,3,4,\text{ or }5$} \sol{$x=5$} 
\bigskip

\ex{$14-3x=20$; \indent $x=-3,-2,-1,0,1,2,\text{ or }3$}
\bigskip

\ex{$x^2-2x=0$; \indent $x=-2,-1,0,1,\text{ or }2$} \sol{$x=0$ and $x=2$}
\bigskip

\ex{$5-y^2=1$; \indent $y=-2,-1,0,1,\text{ or }2$}
\bigskip

Answer each of the following questions based on the information given. Justify your answers.\\

\ex{A technology sales rep quotes the costs for backup data storage systems using the formula $C=50+0.12d$ where $d$ is the number of gigabytes of storage used.  I asked the sales rep to quote me costs for $1000,2000,3000,5000,$ and $10,000$ gigabytes. She wrote up a quote showing a cost of $650$, but the quote doesn’t show the amount of storage that’s based on, and I can’t remember what she used to determine the quote. What amount of storage is the quote based on?
} 
\sol{$5000$ gigabytes}
\bigskip

\ex{My water bill $W$ is calculated using the formula $W=0.02g+5$ where $g$ is the number of gallons used during the month. Last month my bill was $\$25$, but I’m not sure whether I used $500, 1000, 1500,$ or $2000$ gallons. Which of these possibilities, if any, was my actual usage?}
\bigskip

\ex{The mass of carbon dioxide $m$ in grams in a pressurized tank at room temperature is given by the formula $m=44x$ where $x$ is the volume of the tank in liters. A supplier can provide pressurized tanks with volumes of $5,10,20,$ or $50$ liters.  A chemist wrote down that he needed to order a tank with $176$ grams of carbon dioxide, but didn’t write down the volume tank this was based on. What size tank does he need to order?} \sol{20 liters}
\bigskip

\ex{A landscaper estimates that the time required to cut the grass in an athletic field, $h$ (in hours) based on the size of the field, $A$, in square yards is given by the formula $h=\frac{A}{800}+1$. The town has five athletic fields, measuring $1200, 2000, 2800, 3200,$ and $4800$ square yards. He scheduled one of his workers for four and one half hours to cut the grass on one of the town’s fields, but forgot to say which field it was. Which field is she supposed to cut?}
\bigskip

\subsubsection*{Solving Equations Using the Balance Principle}

Solve each of the following equations. If your final answer is a fraction, reduce it to lowest terms.
\begin{tasks}[label={}](2)
	\task\ex{$3x-23=37$} \sol{$x=20$}
	\task\ex{$5x-47=103$}
	\task\ex{$2x+7=32$} \sol{$x=\frac{25}{2}=12.5$}
	\task\ex{$6x+1=13$}
	\task\ex{$9-2z=11$} \sol{$z=-1$}
	\task\ex{$6-5z=36$}
	\task\ex{$3x+2=11+2x$} \sol{$x=9$}
	\task\ex{$7x-3=5x+11$}
	\task\ex{$5x-13=8x+20$} \sol{$x=-11$}
	\task\ex{$7x+2=3x-18$}
	\task\ex{$21x+13=9x-20$} \sol{$x=\frac{-11}{4}=-2.75$}
	\task\ex{$15x-5=61-x$}
	\task\ex{$3(y-2)+2y=14$} \sol{$y=4$}
	\task\ex{$2(y+5)+3y=25$}
	\task\ex{$4(d-1)+3(5-d)=0$} \sol{$d=-11$}
	\task\ex{$6(k+5)-(10+k)=20$}
	\task\ex{$3x+2(3x-5)=7x-4$} \sol{$x=3$}
	\task\ex{$4(2x-1)-2(3x-5)=12$}
\end{tasks}

\bigskip
\ex{$5(3x-1)-2(7x+1)=3(2x+5)-(x+1)$} \sol{$x=\frac{-21}{4}=-5.25$}

\bigskip
\ex{$5t-3(t+1)=4(2t+1)+5$}

\bigskip
\ex{$4h-5(2h-6)=3(h-5)+2(h+3)$}  \sol{$h=\frac{39}{11}=3.5454...$}

\bigskip
\ex{$6(2t-3)-5(3t+1)=3(2t+1)+7(t-3)$}

\bigskip
\ex{A technology sales rep quotes the costs for backup data storage systems using the formula$C=50+0.12d$ where$ d$ is the number of gigabytes of storage used. She wrote up a quote showing a cost of $\$1850$, but the quote doesn’t show the amount of storage that’s based on, and I can’t remember what she used to determine the quote. What amount of storage is the quote based on?}  \sol{$15,000$ gigabytes}

\bigskip
\ex{My water bill$ W$ is calculated using the formula $W=0.02g+5$ where $g$ is the number of gallons used during the month. Last month my bill was $\$37.68$, but I’m not sure how much water I used. How many gallons did I use last month?}

\bigskip
\ex{The mass of carbon dioxide $m$ in grams in a pressurized tank at room temperature is given by the formula $m=44x$ where$x$ is the volume of the tank in liters. A chemist wrote down that he needed to order a tank with $880$ grams of carbon dioxide, but didn’t write down the volume tank this was based on. What size tank does he need to order?} \sol{$20$ liters}

\bigskip
\ex{A landscaper estimates that the time required to cut the grass in an athletic field, $h$ (in hours) based on the size of the field, $A$, in square yards is given by the formula $h=\frac{A}{800}+1$. He scheduled one of his workers for five hours to cut the grass on one of the town’s fields. How many square yards does the field measure?}

\bigskip

\subsubsection*{Solving Equations Involving Fractions}
Solve each of the following equations. If your final answer is a fraction, reduce it to lowest terms.
\begin{tasks}[label={}](2)
	\task\ex{$\frac{1}{3}x-1=\frac{5}{2}$}  \sol{$\frac{21}{2}=10.5$}
	\task\ex{$\frac{1}{4}x-3=\frac{2}{5}$}
	\task\ex{$\frac{3}{8}x+\frac{5}{2}=\frac{5}{6}x-\frac{1}{3}$} \sol{$x=\frac{68}{11}=6.1818...$}
	\task\ex{$\frac{2}{5}x+\frac{9}{10}=\frac{1}{2}x-\frac{3}{4}$}
	\task\ex{$\frac{3}{2}p-1=\frac{5}{8}p-3$} \sol{$p=\frac-{16}{7}$}
	\task\ex{$\frac{2}{3}r+5=\frac{1}{9}r+25$}
	\task\ex{$\frac{y-2}{3}=\frac{y+5}{2}$} \sol{$y=-19$}
	\task\ex{$\frac{x+2}{5}=\frac{x-1}{3}$}
	\task\ex{$\frac{2x-1}{5}+3=\frac{3x}{10}$} \sol{$x=-28$}
	\task\ex{$\frac{4x+3}{20}x=1+\frac{x-1}{5}$}
	\task\ex{$\frac{1}{2}\left(\frac{3}{5}z-\frac{1}{4}\right)=3$} \sol{$z=\frac{125}{12}=10.4166...$}
	\task\ex{$\frac{2}{7}\left(\frac{1}{2}y+\frac{3}{14}\right)=\frac{3}{7}y-1$}

\end{tasks}

\subsubsection*{Identities and Contradictions}

Each of the following equations is either an identity or a contradiction (inconsistent equation).  Determine which.  Justify your answer.

\ex{$3(2x-5)=2(3x-7)$}  \sol{contradiction or inconsistent}

\bigskip
\ex{$5(3x-1)=3(5x+1)$}

\bigskip
\ex{$4x-3(x+5)=7(x+2)-3(2x+1)$} \sol{contradiction or inconsistent}

\bigskip
\ex{$8x-3(2x+1)=5x-3(x+1)$}

\bigskip
\ex{$6x-24=3(2x-8)$} \sol{Identity}

\bigskip
\ex{$6(2x-5)=2(6x-15)$}

\bigskip
\ex{$5(x+2)-(4x-1)=7(x-2)-3(2x-1)$} \sol{contradiction or inconsistent}

\bigskip
\ex{$8(2x-5)-2(3x+1)=3(3x+1)-(7-x)$}

\bigskip

\subsubsection*{Solving for One Variable in Terms of Others}

Solve each of the following equations for the requested variable in terms of the other variable(s).

\begin{tasks}[label={}](2)
	\task\ex{$y=2x-10$ for $x$} \sol{$x=\frac{y+10}{2}$}
	\task\ex{$y=5-3x$ for $x$}
	\task\ex{$P=10+2W$ for $W$} \sol{$\frac{P-10}{2}$}
	\task\ex{$v=80-16t$ for $t$}
	\task\ex{$z=\frac{x-100}{15}$ for $x$} \sol{$x=15z+100$}
	\task\ex{$z=\frac{x-60}{12}$ for $x$}
	\task\ex{$F=\frac{9}{5}K-460$ for $K$} \sol{$K=\frac{5}{9}\left(F+460\right)$}
	\task\ex{$m=\frac{3}{5}t-1$ for $t$}
	\task\ex{$I=PRT$ for $P$} \sol{$P=\frac{I}{RT}$}
	\task\ex{$I=PRT$ for $R$}
	\task\ex{$z=\frac{x-\mu}{\sigma}$ for $x$}  \sol{$x=\mu+z\sigma$}
	\task\ex{$z=\frac{x-\mu}{\sigma}$ for $\mu$}
	\task\ex{$z=\frac{x-\mu}{\sigma}$ for $\sigma$} \sol{$\sigma=\frac{x-\mu}{z}$}
	\task\ex{$E=mc^2$for$m$}
\end{tasks}

\bigskip
\ex{The mass of carbon dioxide $m$ in grams in a pressurized tank at room temperature is given by the formula $m=\frac{44x}{5}$ where $x$ is the volume of the tank in liters. This formula is set up to give the mass based on the volume. A chemist needs a formula which is instead set up to give the volume based on the mass.  Solve the equation for $x$ in terms of $m$ to create the desired formula.}  \sol{$x=\frac{5}{44}m$}

\bigskip
\ex{A landscaper estimates that the time required to cut the grass in an athletic field, $h$ (in hours) based on the size of the field, $A$, in square yards is given by the formula $h=\frac{h}{800}+1$. He wants a formula which gives the size of the field based on the time it takes to mow it.  Solve the equation for $A$ in terms of $h$ to create the formula he wants.}

\bigskip

\subsubsection*{General Exercises}

\ex{Solve $3x-2=5x-8$} \sol{$x=3$}

\bigskip
\ex{Solve $4(5x-1)+3=10(2x+3)$}

\bigskip
\ex{Solve for $z$: $x=3z+2y-5$}  \sol{$z=\frac{x-2y+5}{3}$}

\bigskip
\ex{Solve $5y-1=3y+1$}

\bigskip
\ex{Solve $4(x+2)-3(x+1)=x+5$}  \sol{Identity}

\bigskip
\ex{What is the solution set of the equation $3x+5(x-3)=1$?}

\bigskip
\ex{What is the solution set of the equation $4(2x-5)=8(x-1)$?} \sol{$\emptyset$}

\bigskip
\ex{The daily profit for a tour operator $p$ depends on the number of paying customers taking his tour $t$. The relationship is expressed by the equation $p=65t-350$. Yesterday he made a profit of $\$820$. How many paying customers did he have that day?}

\bigskip
\ex{The daily profit for a tour operator $p$ depends on the number of paying customers taking his tour $t$. The relationship is expressed by the equation $p=65t-350$. This equation is set up to most easily give profit based on the number of paying customers. The tour operator would like to have a formula which is set up to give the number of customers needed based on the amount of profit, so that he can easily figure out how many customers he needs to draw to reach his profit goals. Find such a formula, by solving for $t$ in terms of $p$.} \sol{$t=\frac{t+350}{65}$}

\bigskip
\ex{Solve $4x-3(x-2)=6$}

\bigskip
\ex{Solve $5-2x=9+2x$} \sol{$x=-1$}

\bigskip
\ex{Which, if any, of the values $-2$, $-1$, $0$, $1$, $2$ satisfy the equation $x^2+3x+4=2$?}

\bigskip
\ex{Solve $7=2(1-\frac{1}{8}t)$} \sol{$t=-20$}

\bigskip
\ex{Solve $5x+9=2x-12$}

\bigskip
\ex{Solve $3(2x+5)+4(3+x)=2(4x-1)+2x$} \sol{contradiction or inconsistent}

\bigskip
\ex{Solve for $t$: $m=3t-k$}

\bigskip
\ex{Solve $2z-1=5-\frac{1}{3}z$} \sol{$z=\frac{18}{7}$}

\bigskip
\ex{Solve $12x+10=3x+1+3(3x+3)$}

\bigskip
\ex{What is the solution set of the equation $5(2t-3)+1=2(4t+1)$?}  \sol{$\{8\}$}

\bigskip
\ex{What is the solution set of the equation $3x+9=3(x+3)$?}

\bigskip
\ex{The cost of mobile phone service for the month depends on the number of minutes used. The cost $C$ in dollars, including tax, based on $m$ minutes of usage, is given by the formula $C=1.08(4.95+0.05m)$. If my bill for the month is $\$67.99$ how many minutes did I use?} \sol{$1160$ minutes}

\bigskip
\ex{The cost of mobile phone service for the month depends on the number of minutes used. The cost $C$ in dollars, including tax, based on $m$ minutes of usage, is given by the formula $C=1.08(4.95+0.05m)$. This formula is set up to most easily give the cost based on the number of minutes used. In order to manage my phone use to stay within my budget, I’d like to have a formula which is set up to tell me the number of minutes based on the amount I want my bill to be. Find such a formula, by solving for $m$ in terms of $C$.}

\bigskip
\ex{Solve $6y+3(y+4)=y+12$}  \sol{$y=0$}

\bigskip
\ex{Solve $5(3-x)+20=5x+10$}

\bigskip
\ex{Which, if any, of the values $0$, $1$, $2$, $3$, $4$, $5$ satisfy the equation $x^2-4x= 0$?} \sol{$x=0$ and $x=4$}

\bigskip
\ex{Solve for $c$: $E = \frac{1}{2}c^2$}

\bigskip

\clearpage

%%%%%%%%%%%%%%%%%%%%%%%%%%%%%%%%%%%%%%%%%%%%%%%%%%%%%%%%%%%%%%%%%%%%%%