%%%%%%%%%%%%%%%%%%%%%%%%%%%%%%%%%%%%%%%%%%%%%%%%%%%%%%%%%%%%%%%%%%%%%%
%
% Section: Algebraic Expressions
%
%%%%%%%%%%%%%%%%%%%%%%%%%%%%%%%%%%%%%%%%%%%%%%%%%%%%%%%%%%%%%%%%%%%%%%

\section{Algebraic Expressions}
\label{AlgebraicExpressions}

In the last section, we defined algebraic expressions as any collection of variables and/or numbers tied together by arithmetic operations. In this section, we’ll explore the rules of working with these expressions, and put them to work to our advantage.

We also discussed the idea of using algebraic expressions as a simpler way of expressing mathematical ideas. We observed that instead of saying \quotes{the area of a rectangle is the result of multiplying its length by its width} we could just write the equation \quotes{$A=LW$}. The equation is more concise and lends itself to algebraic substitution when we need to actually calculate areas. This is the goal and the power of algebra: to make things as simple as possible (yes, I know, it often doesn’t feel like it!)

Let’s try another example. Suppose you are working for a landscaping company that installs brick paver patios, and you need to determine the price to quote a job. The total cost of the materials will depend on the area of the patio to be installed. The pavers cost $\$0.97$ per square foot, the crushed gravel for the foundation costs $23$ cents per square foot, and other materials (such as sand) add up to $20$ cents per square foot. The labor for installation costs $\$2.45$ per square foot. In addition, the use of the equipment for the project will cost a flat $\$425$, and your company prices $\$800$ into each job for overhead and profit, with an additional $20$ cents per square foot for overhead and profit as well.

In this scenario, the amount of your quote will depend on just one thing: the square footage of the patio. If we call that $x$, we can see that the pavers will cost $0.97$ times $x$, the gravel $0.20$ times $x$, and so on. So, the quote for an $x$ square foot patio can be expressed algebraically as:
$$0.97x+0.23x+0.20x+2.45x+425+800+0.20x$$

To determine the price for a job, you just would need to plug in the square footage for $x$, and then do the arithmetic. But that’s an awful lot of arithmetic.  There is a simpler way. The costs of pavers, gravel, other materials, labor, and the per-square foot profit and overhead charge all are multiplied by the square footage. Rather than calculating these each individually, we could just add $\$0.97+\$0.23+\$0.20+\$2.45+\$0.20$ to get that the total per square-foot cost is $\$3.85$, and so the part of the quote that depends on the patio’s size is just $3.85x$. Similarly, we can just combine the $\$425$ and $\$800$ overhead and profit cost to get that there is $\$425+\$800=\$1225$ to be added to the quote regardless of the patio’s size. So, putting them all together, the quote could also be expressed algebraically as:
$$3.85x+1225$$

Now, it should be clear that both of these algebraic expressions will give you the same price for the job. But it should also be clear that the second is going to be a lot easier to work with. What we have done here is an example of algebraic simplification. \textbf{Algebraic Simplification}\index{Algebraic Simplification} is the process of finding simpler algebraic expressions that are equivalent to more complicated ones.

Of course, if we want the expressions to give the same results, there must be some rules that we must follow. In this section we will review some of the main rules and techniques of algebraic simplification.

%%%%%%%%%%%%%%%%%%%%%%%%%%%%%%%%%%%%%%%%%%%%%%%%%%%%%%%%%%%%%%%%%%%%%%
%
% Subsection: Algebraic Expressions: Terms
%
%%%%%%%%%%%%%%%%%%%%%%%%%%%%%%%%%%%%%%%%%%%%%%%%%%%%%%%%%%%%%%%%%%%%%%

\subsection{Terms}

Algebraic expressions can be broken down into terms. A term is a part of an algebraic expression that is separated from the other parts by addition (or subtraction, since subtraction is the same as adding a negative.) The number that is multiplied by everything else in a term is called its
coefficient. The \quotes{everything else} is called its literal part.  The following are all examples of terms.

\begin{center}
	\bgroup
	\def\arraystretch{1.5} %add spacing -- cell padding
	\begin{tabular}{c|c|c}
		Term & Coefficient & Literal Part \\
		\hline
		$14x^2y^3$ & $14$ & $x^2y^3$ \\
		$7z$ & $7$ & $z$ \\
		$qrt$ & $1$ & $qrt$\\
		$42$ & $42$ & none 
	\end{tabular}
	\egroup
\end{center}

A couple of things to note from these examples:

\begin{itemize}
	\item Every term has a coefficient. $qrt$ can be rewritten as $1qrt$, and so if there is no coefficient written, it is always assumed to be 1 as in there is one of those terms in the expression.
	\item Not every term has a literal part. A number by itself is still considered a term and is referred to as a \textbf{constant term}.
\end{itemize}

On the other hand, an expression like $3.85x+1225$ is not a term, because the $3.85x$ and the $1225$ are separated from each other by addition. We would say that this expression consists of two terms: $3.85x$ is a term and so is $1225$.

The expression $4x^2y^3-5x+y^4-3$ consists of four terms. While the definition says terms must be separated by addition, subtraction is the same as adding a negative so we can rewrite this as $4x^2y^3+(-5x)+y^4+(-3)$. Usually we don’t actually bother doing this, but we do remember that it could be done. This has an important consequence, though, which the following example illustrates.

\exam{\label{AlgebraicExpressionsExample1}How many terms are there in the expression $2x^3-3x^2+5$?  Identify each term's coefficient and literal part}

\indenttext{There are three terms, separated by the \quotes{$+$} and \quotes{$-$} symbols.

For the first term, $2x^3$, the coefficient is $2$ and the literal part is $x^3$.

For the second term, we need to remember that we could rewrite that subtraction as adding a negative. We don’t need to actually write this, but since the definition of a term requires us to at least think of it this way, we consider the coefficient to be $-3$. (In other words, we treat subtraction as a negative coefficient.) The literal part is $x^2$.

For the third term, the coefficient is $5$, and there is no literal part.
}

\exam{\label{AlgebraicExpressionsExample2}How many terms are there in the expression $\displaystyle \frac{3x}{5y}$? Identify each term’s coefficient and literal part.}

\indenttext{Nothing here can be separated by addition, so this is one term. To determine the coefficient and literal part, remember the rules for multiplying fractions (which in this case we will use to \quotes{unmultiply} this fraction.
$$\frac{3x}{5y}=\frac{3}{5}\frac{x}{y}$$

The coefficient is $\displaystyle \frac{3}{5}$ and the literal part is $\displaystyle \frac{x}{y}$
}

%%%%%%%%%%%%%%%%%%%%%%%%%%%%%%%%%%%%%%%%%%%%%%%%%%%%%%%%%%%%%%%%%%%%%%
%
% Subsection: Algebraic Expressions: Simplifying Terms
%
%%%%%%%%%%%%%%%%%%%%%%%%%%%%%%%%%%%%%%%%%%%%%%%%%%%%%%%%%%%%%%%%%%%%%%

\subsection{Simplifying Terms}

A term can often be written in more than one way. For example, the term
$$5x^3$$

could also be written as
$$5xxx$$

since $x^3$ is just a shorter way of writing $xxx$. But writing $5xxx$ probably seems strange to you – and it should! We almost always would rather write our algebraic expressions in the simplest way we can, and writing out the $x$’s repeatedly is far less simple than just using an exponent.

We could also write this same term as
$$x^3(5)$$

since multiplication is commutative (that is, the order doesn’t matter when you multiply,) But that probably looks weird to you too. We almost always write the coefficient of a term at the start of the term. And because we do this all the time we hardly ever give it any thought.

A term is considered to be in \textbf{simplified form}\index{Algebraic Expression!Simplified} whenever:

\begin{itemize}
	\item the coefficient is written at the start of the term, and
	\item any repeated variables are combined together and written using an exponent.
\end{itemize}

So, while $x^3(5)$, $5xxx$, and $5x^3$ are all perfectly good terms, only $5x^3$ would be considered to be in simplified form.

Now, is the following expression a term?
$$(4xy)(3x^2y)$$

It may look like we have two terms here, but in fact everything in this expression is tied by multiplication. Nothing is being separated by addition (or subtraction), so this is actually just a single term. To see what its coefficient and literal part are, though, we need to do some rearranging, and put this in simplified form. Since everything is multiplied, and multiplication is commutative (order doesn’t matter)
$$(4xy)(3x^2y)=(4)(3)(x)(xx)(y)(y)=12x^3y^2$$

and so we see the coefficient is $12$ and the literal part is $x^3y^2$.

\exam{\label{AlgebraicExpressionsExample3}How many terms are there in the expression $(3x^2)(2x)(x^2)$?  Identify each term’s coefficient and literal part.}

\indenttext{Because everything is tied by multiplication, this is one term. To see the coefficient and literal part clearly, we’ll simplify the term by multiplying.
$$(3x^2)(2x)(x^2)=(3)(2)(1)(xx)(x)(xx)=6x^5$$

So the coefficient is $6$ and the literal part is $x^5$.
}

\exam{\label{AlgebraicExpressionsExample4}How many terms are there in the expression $(3x^2)(2x)-x^2$?  Identify each term’s coefficient and literal part.}

\indenttext{There are two terms here, separated by the subtraction. Simplifying by multiplying we get:
$$(3x^2)(2x)-x^2=(3)(2)(xx)(x)+-1x^2=6x^2+-1x^2$$.

The first term’s coefficient is $6$ and its literal part is $x^3$. The second term’s coefficient is $-1$ and its $x^2$. Remember that the coefficient is negative because if we rewrite the expression with the terms separated by addition.}

What about a term like this?
$$(5x^{35})(3x^{49})(3x)$$

To put this term in simplified form, we can follow the methods used in the previous examples. But we surely don’t want to write out all those $x$’s! In practice, what we do here is to multiply the numbers to get the coefficient, and then count how many $x$’s are being multiplied together to determine the appropriate exponent.  In this case, $5 \times 2 \times 3 = 30$ so the coefficient is $30$. Now, in the first parentheses we have $x^{35}$, so that is the same as $35$ $x$’s multiplied together. In the second parentheses we have $49$ $x$’s, and in the third we get one more. So, in total, there are $35+49+1=85$ $x$’s all together. So, the simplified form of this term is:
$$3x^{85}$$

Notice that if we consider $x$ to be $x^1$, what we’ve done by counting the exponents is the same as just adding up the exponents. This can be stated as the following rule:

%%%%%%%%%%%%%%%%%%%%%%%%%%%%%%%%%%%%%%%%%%%%%%%%%%%%%%%%%%%%%%%%%%%%%%
%
% Definition: Multiplication Rule for Exponents
%
%%%%%%%%%%%%%%%%%%%%%%%%%%%%%%%%%%%%%%%%%%%%%%%%%%%%%%%%%%%%%%%%%%%%%%

\begin{definition}
	\index{Exponents!Multiplication}
	\textbf{\underline{Multiplication Rule for Exponents}}\\
	\bigskip
	When multiplying groupings of the same variable raised to powers, you can just add the powers.  (Make sure to remember though that a plain $x$ is actually treated as $x^1$.)
\end{definition}

\exam{\label{AlgebraicExpressionsExample5}Rewrite the term $(4xy^4)(2x^7)(3x^{12}y^6)$ in simplified form.}

\indenttext{Multiplying the coefficients gives us an overall coefficient of: 
$$4\times 2\times 3 = 24$$

Adding the exponents for $x$ we can see the overall exponent for $x$ will be: 
$$1+7+12=20$$

Likewise, the overall exponent for $y$ will be: 
$$4+6=10$$

So, the simplified form of this term is:
$$24x^{20}y^{10}$$
}

%%%%%%%%%%%%%%%%%%%%%%%%%%%%%%%%%%%%%%%%%%%%%%%%%%%%%%%%%%%%%%%%%%%%%%
%
% Subsection: Algebraic Expressions: Combining Like Terms
%
%%%%%%%%%%%%%%%%%%%%%%%%%%%%%%%%%%%%%%%%%%%%%%%%%%%%%%%%%%%%%%%%%%%%%%

\subsection{Combining Like Terms}

Suppose you need to replace the carpeting in two bedrooms and the hallway of an apartment. One bedroom is $300$ square feet in area, the second is $210$ square feet, and the hallway measures $90$ square feet. How much carpet do you need to replace?

You could say: $300 ft^2 + 210 ft^2 +90 ft^2$. But wouldn’t it be far simpler to just total up the areas and say you needed to replace $600 ft^2$?

Because the three measurements are all measuring the same thing (area to be carpeted) it is both reasonable and simpler to combine them in to a grand total.

We use the same idea with algebraic expressions. While $300x^2+210x^2+90x^2$ is a perfectly good algebraic expression, it would be simpler to just put all those $x^2$ terms together and rewrite the expression as the far simpler $600x^2$. We handle those \quotes{squared $x$‘s} just the same as we handle square feet.

This is a general rule for algebra. We say that two terms are \textbf{like} \index{Like Terms} if and only if they have the same literal parts. Like terms can be combined by adding their coefficients (and keeping the literal part the same).

\exam{\label{AlgebraicExpressionsExample6}Simplify the expression $5x^3-9x^3+7x^3$ by combining like terms.}

\indenttext{Since all three terms have the same literal part ($x^3$) they are all like, and so we combine them by adding their coefficients. Since $5+(-9)+7=3$, the expression simplifies to $3x^3$.
}

Now, let’s change things up a little bit. Suppose that you still need to carpet those two bedrooms, but instead of carpeting the hallway you need to install a runner, which is not sold by area but instead by length. The hallway is $15$ feet long. So, you need to put in $300$ square feet and $210$ square feet of carpeting, and $15$ feet of runner.

So, we need $300 ft^2+210 ft^2+15ft$. Now, it still makes sense to combine the square footage of the two bedrooms. But would it make sense to add $15$ feet to this? Of course not! Area and length measure to different things, and so we can’t reasonably add them together. So, we could say we need $510 ft^2 + 15 ft$, but we would not combine things any further.

Once again, this idea extends to algebraic expressions. If terms are unlike (that is, if they are not like) we cannot combine them at all.

\exam{\label{AlgebraicExpressionsExample7}Simplify the expression $5y^3-2y+3y-11y^3$.}

\indenttext{Since addition is commutative, we can rearrange the $y^3$ terms so they can be added together.
	
Combining the $y^3$ terms gives $5y^3+-11y^3=-6y^3$.

Combining the plain $y$ terms gives $-2y+3y=1y$
	
So, overall we get $-6y^3+y$.

These cannot be combined any further because the two terms in the expression are unlike.
}	
In this example we wrote the $y^3$ term first. There is no reason why we cannot write it the other way around; that is $y-6y^3$ would also be a correct answer. However, we usually write expressions from highest to lowest powered terms. We will follow that convention in this book.

\exam{\label{AlgebraicExpressionsExample8}Simplify the expression $3xy-14x^2y-9xy+4+8x^2y-1$.}

\indenttext{Here we proceed by adding like terms:
\begin{align*}
	3xy-14x^2y-9xy+4+8x^2y-1 & = -14x^2y+8x^2y+3xy-9xy+4+-1\\ 
	&=-6x^2y-6xy+3 
\end{align*}
}
%%%%%%%%%%%%%%%%%%%%%%%%%%%%%%%%%%%%%%%%%%%%%%%%%%%%%%%%%%%%%%%%%%%%%%
%
% Section Algebraic Expressions:  Exercises
%
%%%%%%%%%%%%%%%%%%%%%%%%%%%%%%%%%%%%%%%%%%%%%%%%%%%%%%%%%%%%%%%%%%%%%%

\clearpage

\subsection{Exercises}

\subsubsection*{Terms}

Determine how many terms there are in each of the following algebraic expressions. For each term, state the coefficient and the literal part.

\begin{tasks}[label={}](2)
	\task\ex{$3x^2$} \sol{1 term \\ \hspace*{.3in}coefficient: $3$ \\ \hspace*{.3in}literal part: $x$}
	\task\ex{$4y^3$}
	\task\ex{$4x-3$}\sol{2 terms \\ \indent \hspace*{.3in}coefficients: $4$, $-3$ \\ \indent \hspace*{.3in}literal parts: $x$, none}
	\task\ex{$5-2y$}
	\task\ex{$x^7$} \sol{1 term \\ \indent \hspace*{.3in}coefficient: $1$ \\ \indent \hspace*{.3in}literal part: $x^7$}
	\task\ex{$z$}
	\task\ex{$11$} \sol{1 term \\ \indent \hspace*{.3in}coefficient: $11$ \\ \indent \hspace*{.3in}literal part: none}
	\task\ex{$42$}
	\task\ex{$3x^2$} \sol{1 term \\ \indent \hspace*{.3in}coefficient: $3$ \\ \indent \hspace*{.3in}literal part: $x^2$}
	\task\ex{$11x^5y$}
	\task\ex{$3x+2$} \sol{2 terms \\ \indent \hspace*{.35in}coefficient: $3$, $2$ \\ \indent \hspace*{.35in}literal parts: $x$, none}
	\task\ex{$2y-5$}
	\task\ex{$4z-3w+7$} \sol{3 terms \\ \indent \hspace*{.35in}coefficient: $4$, $-3$, $7$ \\ \indent \hspace*{.35in}literal parts: $x$, $w$, none}
	\task\ex{$a+2xy-3z$}
	\task\ex{$5x^2-3yb+2y^2$} \sol{3 terms \\ \indent \hspace*{.35in}coefficient: $5$, $-3$, $3$ \\ \indent \hspace*{.35in}literal parts: $x^2$, $yb$, $y^2$}
	\task\ex{$3p-4q$}
	\task\ex{$2mn^2+3yz-z$} \sol{3 terms \\ \indent \hspace*{.35in}coefficient: $2$, $3$, $-1$ \\ \indent \hspace*{.35in}literal parts: $mn^2$, $yz$, $z$}
	\task\ex{$x^2-2uv+y^2$}
	\task\ex{$4x^2-2ab+b^2$} \sol{3 terms \\ \indent \hspace*{.35in}coefficient: $4$, $-2$, $1$ \\ \indent \hspace*{.35in}literal parts: $x^2$, $ab$, $b^2$}
	\task\ex{$3x^3+2y^2-5z+1$}
	\task\ex{$2x^4-3c^3+4x^2-x$} \sol{4 terms \\ \indent \hspace*{.35in}coefficient: $2$, $-3$, $4$, $-1$ \\ \indent \hspace*{.35in}literal parts: $x^4$, $c^3$, $x^2$, $x$}
	\task\ex{$2x^3-3x^2+2x-1$}
	\task\ex{$ax^2+by+c$} \sol{3 terms \\ \indent \hspace*{.35in}coefficient: $1$, $1$, $1$ \\ \indent \hspace*{.35in}literal parts: $ax^2$, $by$, $c$}
	\task\ex{$2x^3y+3xy^2-4y^3$}
	\task\ex{$4x^2+3xy-2y^2+z$} \sol{4 terms \\ \indent \hspace*{.35in}coefficient: $4$, $3$, $-2$, $1$ \\ \indent \hspace*{.35in}literal parts: $x^2$, $xy$, $y^2$, $z$}
	\task\ex{$x^4+2x^3-x^2+5x-7$}
	\task\ex{$3s^3+2r^2-4q-1$} \sol{4 terms \\ \indent \hspace*{.35in}coefficient: $3$, $2$, $-4$, $-1$ \\ \indent \hspace*{.35in}literal parts: $s^3$, $r^2$, $q$, none}
	\task\ex{$5xy+3m-2n$}
	\task\ex{$2cd+3ef-4gh+i$} \sol{4 terms \\ \indent \hspace*{.35in}coefficient: $2$, $3$, $-4$, $1$ \\ \indent \hspace*{.35in}literal parts: $cd$, $ef$, $gh$, $i$}
	\task\ex{$x^3+2x^2+3x+4$}
	\task\ex{$4z^3-3z^2$} \sol{2 terms \\ \indent \hspace*{.35in}coefficient: $4$, $-3$ \\ \indent \hspace*{.35in}literal parts: $z^3$, $z^2$}
	\task\ex{$17y^4-3y^2+2y-1$}
\end{tasks}

%%%%%%%%%%%%%%%%%%%%%%%%%%%%%%%%%%%%%%%%%%%%%%%%%%%%%%%%%%%%%%%%%%%%%%
%
% Exercises: Simplifying Terms
%
%%%%%%%%%%%%%%%%%%%%%%%%%%%%%%%%%%%%%%%%%%%%%%%%%%%%%%%%%%%%%%%%%%%%%%

\subsubsection*{Simplifying Terms}

Rewrite each of the following terms in simplified form.

\begin{tasks}[label={}](2)
	\task \ex{ $2x(3x^2)$} \sol{$6x^3$}
	\task \ex{ $4(2y^2)(7y^2)$}
	\task \ex{ $(3x^2)(5x)(2x^4)$} \sol{$30x^7$}
	\task \ex{ $(7y)(2y^5)(3y^2)$}
	\task \ex{ $(3z^3)(4z^4)$} \sol{$12z^7$}
	\task \ex{ $(2p^2)(3p^3)(4p^4)$}
	\task \ex{ $(x^2y^3)(x^5)(3x^4)$} \sol{$3x^11y^3$}
	\task \ex{ $(2p^2)(3p^3q^5)(5q)$}
	\task\ex{$(3xy)(2yz)$} \sol{$6xy^2z$}
	\task\ex{$(4ab)(5cd)$}
	\task\ex{$(2pq^2)(3qr^2)$} \sol{$6pq^3r^2$}
	\task\ex{$(x^2y)(xy^2)$}
	\task\ex{$(3ax)(4by)$} \sol{$12abxy$}
	\task\ex{$(7x^2y)(2xy^3)$}
	\task\ex{$(2ab)(3cd)(4ef)$} \sol{$24abcdef$}
	\task\ex{$(5xy^2)(2x^2z)$}
	\task\ex{$(4ac^2)(3bd)(2ef)$} \sol{$24abc^2def$}
	\task\ex{$(2x^3y^2)(3xyz)$}
	\task\ex{$(xy^2)(x^2y^3)$} \sol{$x^3y^5$}
	\task\ex{$(4a^2b)(3ab^2)(2abc)$}
	\task\ex{$(3x^2y^2)(2xy)(5xyz)$} \sol{$30x^4y^4z$}
	\task\ex{$(2pq^3)(5qr^2)(3rs)$}
	\task\ex{$(x^3)(2x^2y)(3xy^2)(4y^3)$} \sol{$24x^6y^6$}
	\task\ex{$(2a^2b^3)(3ab^2c)(4bc^2)$}
	\task\ex{$(3x^2y^3)(5xy^2z)(2xyz)$} \sol{$30x^4y^6z^2$}
	\task\ex{$(4a^2bc)(3ab^2c^2)(2abc^2)$}
	\task\ex{$(5xy)(2x^2y^2)(3x^3y^3)(4x^4y^4)$} \sol{$120x^10y^10$}
	\task\ex{$(2abc)(3bcd)(4cde)(5def)$}
	\task\ex{$(xy)(x^2y)(xy^2)(x^2y^2)(xy^3)$} \sol{$x^7y^9$}
	\task\ex{$(3a)(4b)(5c)(6d)(7e)(8f)$}
\end{tasks}

%%%%%%%%%%%%%%%%%%%%%%%%%%%%%%%%%%%%%%%%%%%%%%%%%%%%%%%%%%%%%%%%%%%%%%
%
% Exercises: Combining Like Terms
%
%%%%%%%%%%%%%%%%%%%%%%%%%%%%%%%%%%%%%%%%%%%%%%%%%%%%%%%%%%%%%%%%%%%%%%

\subsubsection*{Combining Like Terms}

\begin{tasks}[label={}](2)
	\task\ex{$2x^2-3x+5x+3x^2$} \sol{$5x^2+2x$}
	\task\ex{$5x-7x+3x-2x+x^2+4x+3x^2$}
	\task\ex{$11x^3-3x^2+2x-5+x^2-2x+6$} \sol{$11x^3-2x^2+1$}
	\task\ex{$r^2+2r-r^2-3r+r$}
	\task\ex{$4t-2t+3t-5+t^2-4t^2+2t+7$} \sol{$-3t^2+7t+2$}
	\task\ex{$11j-j^2+3j-9$}
	\task\ex{$3x^2-5x+2x^2-4x+7$} \sol{$5x^2-9x+7$}
	\task\ex{$4y^3+2y^2-3y^3+5y^2-y+10$}
	\task\ex{$2ab+3bc-4ab+2ac-5bc$} \sol{$-2ab+2ac-2bc$}
	\task\ex{$x^2y-xy^2+x^2y^2-2xy+3yx^2$}
	\task\ex{$2p^2q^3-3p^2q^3+5p^2q-pq^3+2p^2q$} \sol{$-p^2q^3+7p^2q-pq^3$}
	\task\ex{$3xy-2yx+4xy-3yx+2yx-xy$}
	\task\ex{$3x^2-5x+2x^2-4x+7$} \sol{$5x^2-9x+7$}
	\task\ex{$4y^3+2y^2-3y^3+5y^2-y+10$}
	\task\ex{$2a+3b-4a+2a-5b$} \sol{$-2b$}
	\task\ex{$x^2y-xy^2+x^2y^2-2xy+3yx^2$}
	\task\ex{$2p^2q^3-3p^2q^3+5p^2q-pq^3+2p^2q$} \sol{$-p^2q^3+7p^2q-pq^3$}
	\task\ex{$y^5-3y^3+2y-y^3+4y^5-5y+3y$}
\end{tasks}

\vspace{12pt}\ex{$3z^2-5+2z-11z+3z^2+4-6z^2+9z+1$} \sol{$0$}

\vspace{12pt}\ex{$3xy-2yx+4xy-3yx+2yx-xy$}

\vspace{12pt}\ex{$2x^3-4x^2y+3xy^2-5x^3+2x^2y-3xy^2$} \sol{$-3x^3-2x^2y$}

\vspace{12pt}\ex{$4a^2b-3ab^2+2a^2b-5ab^2+3ab-4a^2b$}

\vspace{12pt}\ex{$2xy^2-3yx^2+4xy^2-2yx^2+5yx^2-xy^2$} \sol{$5xy^2$}

\vspace{12pt}\ex{$3p^3q-4pq^3+2p^3q-3pq^3+4p^3q-5pq^3$}

\vspace{12pt}\ex{$2a^2b^3-3a^2b^3+4ab^2-2a^2b^3+3ab^2-5ab^2$} \sol{$2ab^2-3a^2b^3$}

\vspace{12pt}\ex{$3x^3y-4xy^3+2x^3y-3xy^3+4x^3y-5xy^3$}

\vspace{12pt}\ex{$4abc-3bca+2cab-5abc+3bac-4cab$} \sol{$-3abc$}

\vspace{12pt}\ex{$2x^4-3x^3y+4x^2y^2-2x^4+3x^3y-5x^2y^2$}

\clearpage

%%%%%%%%%%%%%%%%%%%%%%%%%%%%%%%%%%%%%%%%%%%%%%%%%%%%%%%%%%%%%%%%%%%%%%
