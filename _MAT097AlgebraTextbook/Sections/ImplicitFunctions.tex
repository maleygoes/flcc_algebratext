%%%%%%%%%%%%%%%%%%%%%%%%%%%%%%%%%%%%%%%%%%%%%%%%%%%%%%%%%%%%
%
% Section: Implicit Functions
%
%%%%%%%%%%%%%%%%%%%%%%%%%%%%%%%%%%%%%%%%%%%%%%%%%%%%%%%%%%%%

\section{Implicit Functions}
\label{ImplicitFunctions}

Normally we want our function formulas to be set up so that when we plug in an input value, we simply evaluate the result to get the corresponding output value. For example, the relationship between Fahrenheit and Celsius temperature can be described with the formula

\begin{equation*}
	F=\frac{9}{5}C+32
\end{equation*}

where as you would expect $F$ is the temperature in degrees Fahrenheit and $C$ is the temperature in degrees Celsius. How does this formula work as a function? Is $F$ a function of $C$? Is $C$ as function of $F$? Does it work both ways? Does it work either way?

%%%%%%%%%%%%%%%%%%%%%%%%%%%%%%%%%%%%%%%%%%%%%%%%%%%%%%%%%%%%
%
% Subsection: Implicit Functions: Implicit Functions
%
%%%%%%%%%%%%%%%%%%%%%%%%%%%%%%%%%%%%%%%%%%%%%%%%%%%%%%%%%%%%

\subsection{Implicit Functions}

Now, we can treat the temperature in Fahrenheit as a function of temperature in Celsius (since each Fahrenheit temperature corresponds to one and only one Celsius temperature). Or we can treat Celsius as a function of Fahrenheit (since each Celsius temperature corresponds to one and only one Fahrenheit temperature). And either way, we can use this formula. If you input a value for $C$ into the formula, you can evaluate the result and get the corresponding value of $F$. And, if you input a value of $F$, you can then solve the result to get the corresponding value of $C$. Either way works.\\

However, one way works more nicely than the other. If you treat $C$ as the input, then the formula is perfectly set up to give you $F$ as the output. We say that this equation defines $F$ \textbf{explicitly}\index{Function!Explicit} as a function of $C$. We can even give this function a name and write it in function notation as $f(C)=\frac{9}{5}C+32$.\\

If we treat $F$ as the input, then we have to work through the algebra to figure out what the output value of $C$ would be. It can be done, but it is certainly not anywhere near as convenient as it was the other way around. And, since we don’t have an expression which gives us $C$ explicitly, we can’t use function notation with the formula as given. In this situation, where a function relationship exists but the formula is set up so that we have to solve for the output value when an input is plugged in, we say that the equation defines $C$ \textbf{implicitly}\index{Function!Implicit} as a function of $F$.\\

When a formula defines a function explicitly, that is the more natural way to handle the relationship the formula describes. There are, however, cases where we will have good reason to do things the other way around.

\exam{\label{ImplicitFunctionsExample1} The relationship between my natural gas bill (in dollars) and the amount of gas I use (in therms) is given by the formula $B=12.5+0.85t$ where $B$ is my bill and $t$ is my gas usage.  Is my bill a function of my usage? Is my usage a function of my bill? Which is the more natural way to use this formula?
}

\indenttext{My bill is a function of my usage because for any given amount of gas used I will be billed one and only one amount. Likewise, usage is a function of my bill because for any given bill amount there should be one and only one amount of usage (since if I use more I should be billed more and if I use less I should be billed less – this fits with common sense, and also can be seen from the formula, since a different usage would give a different bill from this formula.)\\

The formula is set up to conveniently take $t$ as the input and give $B$ as the output. So it is more natural to treat my bill as a function of my usage. In other words, this formula explicitly gives my bill as a function of my usage, and implicitly gives my usage as a function of my bill.\\
	
This view also agrees with how things actually happen; you I use a certain amount of gas and then the utility calculates my bill based on this. So the formula is written in a way that agrees with how 	things are actually likely to be done.
}

\exam{\label{ImplicitFunctionsExample2} The formula for the relationship between a sphere’s radius and its volume is usually written $V=\frac{4}{3} \pi r^3$, where $V$ is the volume and $r$ is the radius. Is volume a function of radius? Is radius a function of volume? Which is the more natural way to use this formula?
}

\indenttext{We know that volume is a function of radius because if a sphere with a given radius can only have one volume (you can’t draw a sphere in two different sizes with the same radius.)\\

	Likewise, radius is a function of volume, because a sphere with a given volume can only have one radius (you can’t draw a sphere with a given volume in more than one different way.)\\

	The formula is set up to conveniently take $r$ as an input and give $V$ as the output. So it is more natural to treat volume as a function of radius. In other words, the formula gives volume explicitly as a function of radius, but only implicitly gives radius as a function of volume.
}

\exam{\label{ImplicitFunctionsExample3} The formula for the relationship between the time in seconds and height in meters of a ball thrown upward (with an initial upward velocity of 30 meters per second) can be expressed by the formula $h=-4.9t^2+30t$, where $h$ is the height and $t$ is the time. Is height a function of time? It time a function of height? Which is the more natural way to use this formula?
}

\indenttext{We know that height is a function of time because at any given time the ball will be at one and only one height (it can’t be in two places at once.)\\

	Time is not a function of height, though, because both common sense and experience with similar examples in the past tell us that the ball can be at a given height at two different times (on the way up and on the way down.)\\

	We only have one choice of how to use this formula, with height as a function of time. Fortunately, this formula is set up to conveniently take time as the input and give height as the output. In other words, this formula gives height explicitly as a function of time. It does not give time as a function of height at all.
}

%%%%%%%%%%%%%%%%%%%%%%%%%%%%%%%%%%%%%%%%%%%%%%%%%%%%%%%%%%%%
%
% Subsection: Implicit Functions: Rewriting Implicit Functions in Explicit Form
%
%%%%%%%%%%%%%%%%%%%%%%%%%%%%%%%%%%%%%%%%%%%%%%%%%%%%%%%%%%%%

\subsection{Rewriting Implicit Functions in Explicit Form}

Sometimes we have an equation that does not define any function explicitly, but nonetheless expresses a relationship between quantities. For example, suppose a refiner is making $500$ gallons of a fuel blend of gasoline and ethanol. If we let $G$ represent the amount of gasoline (in
gallons) and let $E$ represent that amount of ethanol (in gallons), we can see that

\begin{equation*}
	G+E=500.
\end{equation*}

Is the amount of gasoline a function of the amount of ethanol? Yes! We can see this both from common sense (if I know how much ethanol was used, there is only one possible amount of gasoline that could have been used to bring the total to 500 gallons), and from algebra (if you plug in any given value for $E$ you can solve the equation to get one and only one value for $G$.\\

Is the amount of ethanol a function of the amount of gasoline? Again, the answer is yes, for pretty much exactly the same reasons.\\

This formula, though, does not define either of these functions explicitly. Whether we input gasoline or ethanol, we will have to do some algebra to get the other quantity. We can however create an explicit function from this formula by solving for whichever variable we want to be output in terms of the variable we want to be input.\\

Suppose we want to treat the amount of ethanol as a function of the amount of gasoline. So we want a formula that is set up to give $E$ explicitly – a formula like this:

\begin{equation*}
	E = something \text{ } based \text{ } on \text{ } G 
\end{equation*}

So we start with the given equation, and solve it for $E$:

\begin{align*}
	G+E&=500\\
	E&=500-G
\end{align*}

So we now have a formula that gives ethanol as a function of gasoline explicitly!\\

It’s not hard to see that we could also get gasoline as a function of ethanol explicitly with the formula $G=500-E$.

\exam{\label{ImplicitFunctionsExample4} The equation $2x+3y=6-2x$ implicitly defines $x$ as a function of $y$ and also implicitly defines $y$ as a function of $x$. Rewrite this equation to give these functions explicitly.
}

\indenttext{First, to find $x$ as a function of $y$ means we want an equation that gives $x$ as the output. So we solve for $x$:
	\begin{align*}
		2x+3y&=6-2x\\
		4x+3y&=6\\
		4x&=6-3y\\
		x&=\frac{6-3y}{4}
	\end{align*}
	We can rewrite this in function notation if we want as $x=f(y)=\frac{6-3y}{4}$.\\
	\newline
	Now to find $y$ as a function of $x$ we solve for $y$:
	\begin{align*}
		2x+3y&=6-2x\\
		3y&=6-4x\\
		y&=\frac{6-4y}{3}
	\end{align*}
	We can rewrite this in function notation also as $y=g(x)=\frac{6-4y}{3}$.
}

\bigskip

In this last example we were told that the equation defined functions implicitly. What if we aren’t told that? How can we tell whether or not an equation contains an implicit function? \\

When the equation comes from a mathematical model, the meaning of the model often tells us this. In the gasoline-ethanol example, even without seeing the formula you could have figured that each would have to be a function of the other. In Example \ref{ImplicitFunctionsExample3} it made sense that time would not be a function of height because, unless we assume the baseball is thrown so hard that it escapes the Earth’s gravitation and continues on to explore the universe, the ball will both go up and come down.\\

But we can’t always rely on this. Sometimes we don’t know the purpose behind the equation we are working with, and sometimes the equation models something we don’t know enough about to say whether or not a function exists. How can we tell in those cases?\\

We can determine whether an equation implicitly defines a function by trying to explicitly find a formula for the function. In the previous example (Example \ref{ImplicitFunctionsExample4} you could have seen that $x$ was a function of $y$ by solving for $x$ in terms of $y$. Because you can algebraically modify the equation to be an explicit function, you know the function exists.

\exam{\label{ImplicitFunctionsExample5} Suppose $\frac{t-3}{2}=\frac{p+4}{5}$.  Is $t$ a function of $p$?}

\indenttext{If we can solve for $t$ in terms of $p$ explicitly, then it must be. So we’ll try doing that.
	\begin{align*}
		\frac{t-3}{2}&=\frac{p+4}{5}\\
		10*\left(\frac{t-3}{2}\right)&=10*\left(\frac{p+4}{5}\right)\\
		5(t-3)&=2(p+4)\\
		5t-15&=2p+8\\
		5t&=2p+23\\
		t&=\frac{2p+23}{5}
	\end{align*}
	Since we can find an explicit formula for $t$ as a function of $p$ from this equation, we can conclude that $t$ is indeed a function of $p$.
}

\bigskip

We could also have found that $p$ is a function of $t$ in this example in much the same way. Since that’s pretty similar we won’t show the work for that here, but you may want to try to work that through yourself now.

\exam{\label{ImplicitFunctionsExample6} Suppose $3h=k^2-11$. Is $h$ a function of $k$? Is $k$ a function of $h$?}

\indenttext{We try to solve for $h$ first. That’s actually pretty easy:
	\begin{align*}
		3h&=k^2-11\\
		h&=\frac{k^2-11}{3}
	\end{align*}
	So since we could get an explicit formula for $h$ in terms of $k$ we can say that $h$ is indeed a function of $k$.\\
	\newline
	On the other hand, we run into trouble when we try to solve for $k$:
	\begin{align*}
		3h&=k^2-11\\
		3h+11&=k^2\\
		k^2&=3h+11
	\end{align*}
	To get rid of the exponent, we need to take the square root of both sides of this equation.  However, remember that every positive number has two square roots - one positive and one negative. So we get:
	\begin{equation*}
		k=\sqrt{3h+11} \text{    OR    } k=-\sqrt{3h+11}
	\end{equation*}
	This equation can be written compactly as:
	\begin{equation*}
		k=\pm \sqrt{3h+11}
	\end{equation*}
	The symbol \quotes{$\pm$} is used to indicate \quotes{plus or minus}. Now, while this is a formula for $k$, it is a formula that gives two different results. So, for a given input value of $h$, this formula can give two different outputs for $k$. For this reason, we can conclude that $k$ is not a function of $h$. 
}

So, in summary, if we can find an explicit formula (one that does not involve anything like \quotes{$\pm$} which gives multiple answers), we have a function. If we try to solve and end up with an ambiguous formula (such as the one we just found), we do not have a function.\\

There might be times, though, where you try to solve for a variable and are simply unable to do it. Returning to Example \ref{ImplicitFunctionsExample3}, where we had the formula $h=-4.9t^2+30t$, we determined based on the meaning of the formula that $t$ is not a function of $h$. If you try to solve this equation for $t$ in
terms of $h$ you probably won’t meet with much success. Could you also conclude that it is not a function from an inability to solve for $t$?\\

The answer is emphatically NO! Just because you can’t solve an equation with the algebra tools you’ve learned so far, does not necessarily mean it can’t be done. \quotes{I can’t do it} doesn’t mean \quotes{it can’t be done!} By itself, inability to solve does not tell you one way or the other whether or not the relationship is a function. In the case of Example \ref{ImplicitFunctionsExample3}, we can’t solve for $t$ and we know for other reasons that $t$ is not a function of $h$.\\

But other times we may not be able to solve and the relationship may still be a function. For example, the equation $p-10=2^t$ implicitly defines $p$ as a function of $t$ (that’s not hard to show), but it also defines $t$ as a function of $p$ (though solving this equation for $t$ requires tools that lie far
beyond the scope of this course.) To be able to say that $t$ is a function of $p$ here would require either those more advanced tools, or knowing something about what $p$ and $t$ represent.\\

In the exercises, you won’t be given any questions where whether or not the relationship is a function cannot be determined. In all the problems you will see in this course, it will be possible to tell whether or not the relationship is a function either using algebra we are already familiar with, or by thinking about what the variables in the equation represent.\\

%%%%%%%%%%%%%%%%%%%%%%%%%%%%%%%%%%%%%%%%%%%%%%%%%%%%%%%%%%%%
%
% Subsection: Implicit Functions: Switching Input and Output Variables
%
%%%%%%%%%%%%%%%%%%%%%%%%%%%%%%%%%%%%%%%%%%%%%%%%%%%%%%%%%%%%

\subsection{Switching Input and Output Variables}

Let’s return now to the Fahrenheit-Celsius example we used to start this chapter. We saw that the formula $F=\frac{9}{5}C+32$ gives Fahrenheit explicitly as a function of Celsius. We also noted that this formula also gives Celsius as a function of Fahrenheit \underline{implicitly}.\\

If we want to input Celsius and get Fahrenheit as an output we are fine with the formula as it is, but if we want to go the other way this formula makes that possible, but tedious. It is almost always more convenient to have a formula that gives the function you want \underline{explicitly}. We can accomplish this by solving for the output variable we want.

\exam{\label{ImplicitFunctionsExample7} The formula $F=\frac{9}{5}C+32$ implicitly gives $C$ as a function of $F$. Find a formula that gives $C$ as a function of $F$ explicitly.
}

\indenttext{Solving for $C$:
	\begin{align*}
		F&=\frac{9}{5}C+32\\
		F-32&=\frac{9}{5}C\\
		5(F-32)&=5\left(\frac{9}{5}C\right)\\
		5F-160&=9C\\
		C&=\frac{5F-160}{9}
	\end{align*}
}

Functions that express the same relationship with their input and output variables switched are called inverses of each other. We won’t be doing much with inverse functions in this course, but they will be very important in later mathematical coursework.

%%%%%%%%%%%%%%%%%%%%%%%%%%%%%%%%%%%%%%%%%%%%%%%%%%%%%%%%%%%%
%
% Subsection: Implicit Functions: Exercises
%
%%%%%%%%%%%%%%%%%%%%%%%%%%%%%%%%%%%%%%%%%%%%%%%%%%%%%%%%%%%%

\clearpage

\subsection{Exercises}

\subsubsection*{Implicit Functions:}
\ex{In the formula $r = 5s - 3$ which is the more natural choice for the input variable, and which is the more natural choice of output variable?}
\sol{input $s$, output $r$}

\bigskip
\ex{In the formula $p = 7t^2 - 5t - 4$ which is the more natural choice for the input variable, and which is the more natural choice of output variable?}

\bigskip
\ex{In the formula $\frac{1}{2}v^2 - 2v = E$ what would be the most natural choices for input and output variable?}
\sol{input $v$, output $E$}

\bigskip
\ex{In the formula $k^3 = z$ what would be the most natural choices for input and output variable?}

\bigskip
\ex{The formula which relates tuition to credit hours taken for part-time students at Falstaff Community College is given by the formula $T = 180 + 375c$, where $T$ is the tuition and $c$ is the number of credit hours. Is tuition a function of credit hours? Is credit hours a function of tuition? Which is the more natural way to use this formula? Explain your reasoning.}
\sol{Tuition is a function of credit hours.  This makes sense because the tuition is based on and determined by the number of credit hours, and also because the equation is set up to use $c$ as the input.}

\bigskip
\ex{The amount of water (in gallons) a backyard pool contains is related to the depth of the water (in feet) by the formula $W = 2000d$, where $W$ is the water volume and $d$ is the depth. Is the water volume a function of depth? Is depth a function of volume? Which is the more natural way to use this formula? Explain your reasoning.}

\bigskip
\ex{The hours of daylight, $D$ in Ithaca, New York $t$ days after the Spring Equinox is approximated by the formula $D=12+\frac{18}{365}t-\frac{38}{49000000}t^3+\frac{1}{32000000000}x^5$.  Is $D$ a function of $t$?  Is $t$ a function of $D$?  Which is the more natural way to look at this relationship? Justify your answers.}
\sol{$D$ is a function of $t$.  The time of year determines how many hours of daylight, and the formula is set up with $t$ as the input.}

\bigskip
\ex{The length in centimeters, $L$, of a yoyo string $s$ seconds after I release the yoyo from my hand is given by the formula $L=12s(4-s)$. Is $L$ a function of $s$? Is $s$ a function of $L$? Which is the more natural way to look at this relationship? Justify your answers.}

\subsubsection*{Rewriting Implicit Functions in Explicit Form}

\ex{The equation $6y - 5x = 30$ implicitly defines $y$ as a function of $x$. Rewrite this equation to give this function explicitly.}
\sol{$y=\frac{5x+30}{6}$}

\bigskip
\ex{The equation $11y - 2t = 88$ implicitly defines $y$ as a function of $t$. Rewrite this equation to give this function explicitly.}

\bigskip
\ex{The equation $2r - 3p = 6 + r$ implicitly defines $r$ as a function of $p$. Rewrite this equation to make the function explicit.}
\sol{$r=3p+6$}

\bigskip
\ex{The equation $5j + 7q = 100 + 2q$ implicitly defines $q$ as a function of $j$. Rewrite this equation to make the function explicit.}

\bigskip
\ex{In the equation $5(y - 4) + 2x = 8y$, the variable $x$ is a function of $y$ implicitly. Rewrite this equation so that this function is explicit.}
\sol{$x=\frac{3y+20}{2}$}

\bigskip
\ex{In the equation $2(x + 1) - 3(y + 2) = 4x + y$, the variable $x$ is implicitly a function of $y$. Rewrite this equation so that this function is explicit.}

\bigskip
\ex{In the equation $\frac{m+3n}{2}=\frac{m-4}{6}$, is $m$ a function of $n$? Justify your answer.}
\sol{Yes, it is.  The equation can be solved for $m$ in terms of $n$}

\bigskip
\ex{In the equation $\frac{m + 3n}{2} = \frac{m - 4}{6}$, is $n$ a function of $m$? Justify your answer.}

\bigskip
\ex{In the equation $\frac{2t}{3} - \frac{s}{2} = \frac{t - 2}{5}$, is $s$ a function of $t$? Justify your answer.}
\sol{Yes, it is.  The equation can be solved for $s$ in terms of $t$}

\bigskip
\ex{In the equation $\frac{2t}{3} - \frac{s}{2} = \frac{t - 2}{5}$, is $t$ a function of $s$? Justify your answer.}

\bigskip
\ex{
	\begin{enumerate}[label=(\alph*)]
		\item In the equation $x^2 + y = 5$, is $y$ a function of $x$? Justify your answer.
		\item In the same equation, is $x$ a function of $y$? Justify your answer.
	\end{enumerate}
}
\sol{a. Yes, it is.  The equation can be solved for $y$ in terms of $x$\\
	b.  No, it is not.  If we tried to solve for $x$, we'd have to take the square root of both sides in the process, which would introduce the $\pm$ problem discuss in the chapter.}

\bigskip
\ex{
	\begin{enumerate}[label=(\alph*)]
		\item Suppose $3x^2 = 5y^2 - 8$. Is $y$ a function of $x$? How do you know?
		\item In the same equation, is $x$ a function of $y$? How do you know?
	\end{enumerate}
}

\subsubsection*{Switching Input and Output Variables}

\bigskip
\ex{The equation $w = 80 - 2l$ gives $w$ as a function of $l$. Rewrite this equation to make $l$ an explicit function of $w$.}
\sol{$l=\frac{80-w}{2}$}

\bigskip
\ex{Rewrite $t = \frac{3p - 2}{5}$ to give $p$ as an explicit function of $t$.}

\bigskip
\ex{The relationship between my natural gas bill (in dollars) and the amount of gas I use (in therms) is given by the formula $B = 12.50 + 0.85t$ where $B$ is my bill and $t$ is my gas usage. This formula gives my bill as a function of my usage. Rewrite this formula so that it gives my usage as a function of my bill.}
\sol{$t=\frac{B-12.5}{0.85}$}

\bigskip
\ex{The amount of music (in hours) that can be stored on a digital music player is a function of the size of its memory capacity (in megabytes), and is given by the formula $t = \frac{x-120}{10}$. Is memory capacity also a function of amount of music? If so, rewrite this equation in a way that gives this function explicitly.}

\subsubsection*{Grab Bag}

\bigskip
\ex{
	In which of the following equations is $y$ a function of $x$? How do you know?
	\begin{enumerate}[label=(\alph*)]
		\item $x - 2y = 3x - y + 2$
		\item $x^2 - 3y = 5 - 2y$
		\item $11x - 3y^2 = 18 - y^2$
		\item $\frac{4 - 3x}{2} - \frac{2y + 7}{3} = 0$
		\item $y - 2(y - 3) = x$
	\end{enumerate}
}
\sol{a. Function, can be solved for $y$.\\
	b. Function, can be solved for $y$.\\
	c. Not a function, getting rid of squared term would result in $\pm$ at some point.\\
	d. Function, can be solved for $y$.\\
	e. Function, can be solved for $y$.
}

\bigskip
\ex{
	Suppose $2T = 3(P - 7)$.
	\begin{enumerate}[label=(\alph*)]
		\item Is $T$ a function of $P$? If so, find an explicit formula. If not, explain why not.
		\item Is $P$ a function of $T$? If so, find an explicit formula. If not, explain why not.
	\end{enumerate}
}

\bigskip
\ex{
	A club is planning a field trip to Washington D.C. The total cost of the trip is a function of the number of members who will be going, and is given by the formula $C(x) = 1850 + 475x$.
	\begin{enumerate}[label=(\alph*)]
		\item Let $K$ be a variable which represents the cost of the trip. Rewrite this function formula as an equation which gives $K$ as a function of $x$.
		\item Is the number of members going a function of total cost? If so, rewrite this equation so that it gives members going as a function of cost explicitly. If not, explain why not.
	\end{enumerate}
}
\sol{a.  $K=1850+475x$\\
	b.  It is, $x=\frac{K-1850}{475}$}

\bigskip
\ex{
	A movie theater operator has hired a business consultant to help determine price strategy. The consultant developed a model which projects the theatre’s monthly profits as a function of the regular adult ticket price. Suppose her model uses the function $p(t) = 800(-t^2 + 20t - 64)$.
	\begin{enumerate}[label=(\alph*)]
		\item Rewrite this function formula as an equation with the variable $R$ used to represent projected monthly profit.
		\item Graph the function in the window $0 \leq t \leq 20$ and $0 \leq R \leq 40000$.
		\item With the algebra we know so far, it is not possible to solve the equation for $t$. That only means that we can’t find a formula which gives $t$ as an explicit function of $R$; that does not tell us whether or not ticket price is a function of projected profit. Use the graph to determine whether or not ticket price is a function of projected profit, and explain your reasoning.
	\end{enumerate}
}

\bigskip
\ex{
	A refiner has 1000 gallons of pure gasoline which she is blending with ethanol. The relationship given between the percent of ethanol in the blend, $P$, and the gallons of ethanol she adds, $g$, is given by the formula $P = \frac{100g}{1000 + g}$. This formula gives the percent explicitly as a function of gallons added. Is the amount of added ethanol a function of the percent of ethanol in the blend? How do you know? (You may, or may not, choose to use the formula to answer this question.)
}
\sol{It is a function, because common sense tells you that to get a given percentage of alcohol there would be only one answer to the question of how much ethanol you would need to add.  It is also possible to solve for $g$ in terms of $P$.}

\bigskip
\ex{
	Find the inverse function for $f(x) = 3x - 7$.
}

\clearpage
