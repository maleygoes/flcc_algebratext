
%%%%%%%%%%%%%%%%%%%%%%%%%%%%%%%%%%%%%%%%%%%%%%%%%%%%%%%%%%%%%
%
% Section: Finding Formulas for Linear Functions
%
%%%%%%%%%%%%%%%%%%%%%%%%%%%%%%%%%%%%%%%%%%%%%%%%%%%%%%%%%%%%

\section{Finding Formulas for Linear Functions}
\label{FormulaLinearFunctions}

Of course not every function is linear, but lots and lots of useful functions are. We’ve already seen some of the properties of linear functions, specifically how to find their slopes and intercepts, and what those slopes and intercepts tell you.\\

In almost all of our work so far, though, we’ve assumed that we already had a formula for a given linear function, and then could work from that formula. In this section, we’ll fully explore something we’ve hinted at in a few examples but not yet developed: how to find the formula for a function that we’ve determined to be linear. As you’ll see, the properties of linear functions that we’ve already observed are not just helpful in interpreting functions whose formulas we somehow already know – they are incredibly useful in helping us find formulas for those functions in the first place!

%%%%%%%%%%%%%%%%%%%%%%%%%%%%%%%%%%%%%%%%%%%%%%%%%%%%%%%%%%%%%
%
% Subsection: Finding Formulas for Linear Functions: Formulas for Linear Functions Given Slope and Intercept
%
%%%%%%%%%%%%%%%%%%%%%%%%%%%%%%%%%%%%%%%%%%%%%%%%%%%%%%%%%%%%

\subsection{Formulas for Linear Functions Given Slope and Intercept}

In Chapter \ref{LinearFunctionsandIntercepts}, we introduced the slope-intercept form for a linear function, $f(x)=mx+b$. We noted that a function is linear if and only if it can be written in this form. So, if we know the slope and vertical intercept of a linear function, we can write a formula for the function simply by plugging that slope and intercept into this general form.

\exam{\label{FormulaLinearFunctionsExample1} Suppose $f(x)$ is a linear function whose slope is $-5$ and vertical intercept is $(0, 8)$.  Find a formula for $f(x)$.}

\indenttext{Since it is linear, we know that $f(x)=mx+b$ for the appropriate values of $m$ and $b$. Now $m$ is the slope, so we know that $m$ is $-5$. The vertical intercept can be given as a point, as it is here, or just as the output value of that point which would be just $8$, and so that’s $b$. So substituting those
	values into the slope-intercept form we get that 
	$$f(x)=-5x+8$$
}

\bigskip

It really is just that simple, if we know the slope and vertical intercept. Of course, we won’t always have those values at our disposal, but since they are pretty significant features, it’s actually not all that unusual to have them available to find a linear function’s formula. In fact, when we worked out a couple formulas at the start of Section \ref{LinearFunctionsandIntercepts} using common sense reasoning, this is in fact what we were doing.

\exam{\label{FormulaLinearFunctionsExample2} A new bottle of vitamins contains $75$ pills, and a full day’s dose is $3$ pills. Find a formula for the number of pills the bottle will contain after $t$ full daily doses have been taken.}

\indenttext{The function is linear, because with each full daily dose the number of pills changes by the same amount, it drops by three. The vertical intercept is the starting point, or more precisely, it is the number of pills in the bottle when the number of daily doses taken is zero. So $m$ is $-3$ and $b$ is $75$.  And so:  $$f(t)=-3t+75$$
}

\bigskip

It’s not unusual to change the formula a bit, especially when a different order might seem to better reflect the \quotes{story} the function is describing. Here, we start with $75$ pills and then take away $3$ each day, so it might seem more natural to write this as $f(t)=75-3t$. Of course, this is algebraically
equivalent to the formula we wrote above, so this alternative is also a correct formula for the function. At least for now, we will try to stick with the $mx+b$ 	order, to emphasize that this is what we are using, but it is not at all incorrect to write this function either way.

%%%%%%%%%%%%%%%%%%%%%%%%%%%%%%%%%%%%%%%%%%%%%%%%%%%%%%%%%%%%%
%
% Subsection: Finding Formulas for Linear Functions: Formulas for Linear Models Given Slope and Input-Output Pair
%
%%%%%%%%%%%%%%%%%%%%%%%%%%%%%%%%%%%%%%%%%%%%%%%%%%%%%%%%%%%%

\subsection{Formulas for Linear Models Given Slope and Input-Output Pair}

Of course, we aren’t always lucky enough to have both the slope and the vertical intercept. If we have the slope and one input-output pair for a linear function, we can still find the function’s formula, albeit with a little more effort required. We will present two different methods for doing this by way of a specific example. Either method will always work, and so which one you use is a matter of preference.\\

\begin{definition}
	\textbf{\underline{Method \#1: Back Substitution}}\\
	\vspace{12 pt}
	\begin{enumerate}
		\item Start with the general form $f(x)=mx+b$ (replacing the $x$ and $f(x)$ with the actual names of the function and input variable if they are different)
		\item Plug in the known value of the slope as $m$.
		\item Using the known input-output pair, plug in the input for $x$ and the output for $f(x)$
		\item Solve the resulting equation for $b$ to get the vertical intercept.
		\item You now have the slope and vertical intercept; use these in the \quotes{$f(x)=mx+b$} form to get the function formula.
	\end{enumerate}
\end{definition}

We can see how this works with the following two examples. We will actually start with a \quotes{numbers only} example just to show how this method works, and then will move to a real-world application example which will allow us to see both how the method works and a little bit of interpretation along the way.

\exam{\label{FormulaLinearFunctionsExample3} Find the formula for the linear function $g(t)$ given that the function’s slope is $-4.5$ and the function’s graph passes through the point $(5, 12)$.}

\indenttext{
	\begin{enumerate}
		\item We begin with the fact that since we know $g(t)$ is linear, it must have a formula of the form:  $$g(t)=mx+b$$
		\item Now since we know the slope, we can plug that in to get:  $$g(t)=-4.5t+b$$
		\item Next, since the graph passes through the point $(5,12)$ we know that $g(5)=12$, and so if we plug in the input of $5$ for $t$ the output must be $12$. Plugging those values in we get: $$12=-4.5(5)+b$$
		\item Now this is an equation we can solve for $b$. Doing that we get:  $$12 = -22.5 + b$$  $$b=34.5$$
		\item So now we know that our function’s slope is $-4.5$ and its vertical intercept is $34.5$. So, putting these facts together, we conclude that the formula is:  $$g(t)=-4.5t+34.5$$
	\end{enumerate}
}

\bigskip

In this first example we listed the numbered steps of the method explicitly. We won’t do that in future examples, but you can see that we are in fact still following them. In the next example, we’ll use the method to find a formula where there is a real world interpretation to the linear function and see how this gives some extra insight into why this method works.

\exam{\label{FormulaLinearFunctionsExample4} An online music site charges a fixed monthly membership fee, which allows unlimited streaming. Members can also download songs for $\$0.85$ each. Last month I downloaded $32$ songs and paid $\$35.20$.  Find a formula for the amount I pay as a function of the number of songs I download.}

\indenttext{First we note that this function would indeed be linear. The input is music purchased (counted in songs), and the output is cost (in dollars). The rate of change would be dollars per song, and we are told this is a constant $\$0.85$ per song. So the function is linear.\\
	\newline
	If we let $s$ be the number of songs and $C(s)$ be the cost for downloading that many songs, then:  $$C(s)=mx+b$$
	Plugging in the known slope we get:  $$C(s)=0.85s+b$$
	Since an input of $32$ songs gives and output of a $\$35.20$ charge, we know: $$35.2 = 0.85(32)+b$$
	And so we solve for $b$ and get:  $$35.2 = 27.2+b$$  $$b=8$$
	Putting the slope and vertical intercept together we get that the overall formula must be:  $$C(s)=0.85s+8$$
	We can add some interpretation to this formula as well. The vertical intercept is the charge you’d pay when the input - the number of songs you download - is zero. So, we can interpret this function as telling us that you pay $\$8$ per month just to have the service, plus $\$0.85$ per download.
}

\bigskip

This example agrees with a less-structured but commonsense way of figuring this out. Rather than using the formal method we’ve laid out here, you could have figured this formula out by reasoning that since each download costs $\$0.85$, I must have paid $32(\$0.85)=\$27.20$ for the downloads.  Since my total charge was $\$35.20$, the membership fee must have been $\$35.20-\$27.20 = \$8$.  From this you could have easily arrived at the formula.\\

Such reasoning is perfectly correct, and in fact if you look through the example, it’s actually pretty much what we did! There’s nothing wrong with taking this less-structured approach, but in cases where the interpretation is not so easy to see it may not work so well. The structured method we’ve laid out here always works, whether the interpretation is easy, difficult, or even if there is no interpretation at all (as in Example \ref{FormulaLinearFunctionsExample3}).\\


Now at the start of this section we promised a second method, and we will introduce that now, and then illustrate its use with the same two examples we’ve just worked out with the first method.  Again, it must be emphasized that both methods will always work, but for a number of reasons, your instructor may prefer, or even insist that you use one or the other. Make sure that you are clear about the expectations in the course that you are taking.

%%%%%%%%%%%%%%%%%%%%%%%%%%%%%%%%%%%%%%%%%%%%%%%%%%%%%%%%%%%%%
%
% Subsection: Finding Formulas for Linear Functions: Point-Slope Form
%
%%%%%%%%%%%%%%%%%%%%%%%%%%%%%%%%%%%%%%%%%%%%%%%%%%%%%%%%%%%%

\subsection{Point-Slope Form}

This second method is a formula, derived from the formula for slope. Before presenting the formula, we’ll briefly show where it comes from. Though you need not fear being asked to reproduce the derivation yourself, it’s good to see at least once that our formulas really do come from somewhere!\\

Our formula for slope \index{Linear Function!Slope} can be written in the form:  

$$m=\frac{\Delta f(x)}{\Delta x} = \frac{\Delta output}{\Delta input}$$

Now, suppose we have a particular input-output pair for a linear function. If we call this particular input value $x_0$ then we would call the output value $f(x_0)$ or using an output variable we could call it just $y_0$. To keep the notation a little simpler, we will use $y_0$ instead of the function notation here.\\

Now let’s let just plain $x$ and just plain $y$ be any other input and output for this function. (To be clear: $(x_0,y_0)$ is the specific input-output pair that we know, and just plain $(x,y)$ represents any other input-output pair for the function - not any specific input-output pair that we know.)\\

$\Delta output$ and $\Delta input$ can be found by subtracting outputs and inputs, and so going back to the slope formula we’d get:  $$m=\frac{y-y_0}{x-x_0}$$
Which we can then solve for $y$ by first clearing the fraction and then simplifying:

\begin{align*}
	\frac{y-y_0}{x-x_0}&=m\\
	\left( \frac{y-y_0}{x-x_0} \right)(x-x_0) &= m(x-x_0)\\
	y-y_0&=m(x-x_0)\\
	y&=m(x-x_0)+y_0
\end{align*}

This work has led us to the point-slope form for the equation of a linear function. Plugging in the slope and known input-output pair directly into this form gives an equation for the linear function; it can then be readily simplified into slope-intercept form.

\begin{definition}
	\textbf{\underline{Method \#2: Point-Slope Form}}\\
	\index{Linear Function!Point-Slope Form}
	\bigskip
	If $m$ is the slope and $(x_0,y_0)$ is a given input-output pair for a linear function, the function’s formula can be found by plugging these values into the form:  $$y=m(x-x_0)+y_0$$
	If the input variable is a letter other than $x$ that letter would be used in place of $x$ in this form; if there is an output variable given then that letter would likewise be used in place of $y$.\\
	\bigskip
	This can then be simplified into slope-intercept form if desired, and the $y$ can be replaced with an $f(x)$ to use function notation.
\end{definition}

We will illustrate the use of this method by reworking the previous two examples.

\exam{\label{FormulaLinearFunctionsExample5} Find the formula for the linear function $g(t)$ given that the function’s slope is $-4.5$ and the function’s graph passes through the point $(5,12)$.}

\indenttext{Here the input variable is $t$ and so we use that in place of $x$. There is no output variable given so we will use $y$.\\

	We have here that $m=-4.5$, $t_0=5$ and $y_0=12$.  We plug these into the point-slope form to get:

	\begin{equation*}
		y=m(t-t_0)+y_0\\
		y=-4.5(t-5)+12
	\end{equation*}

	and now we simplify:

	\begin{equation*}
		y=-4.5t+22.5 + 12\\
		y=-4.5t+34.5
	\end{equation*}

	and finally rewriting in function form by replacing the $y$ with an $g(t)$:
	$$g(t)=-4.5t+34.5$$
}

\exam{\label{FormulaLinearFunctionsExample6} An online music site charges a fixed monthly membership fee, which allows unlimited streaming. Members can also download songs for $\$0.85$ each. Last month I downloaded $32$ songs and paid $\$35.20$. Find a formula for the amount I pay as a function of the number of songs I download.}

\indenttext{Just as we did the first time, we must first note that this function would indeed be linear. The input is music purchased (counted in songs), and the output is cost (in dollars). The rate of change would be dollars per song, and we are told this is a constant $\$0.85$ per song. So the function is linear.\\

	If we let $s$ be the number of songs and $C(s)$ be the cost for downloading that many songs, as we did before, and let $y$ stand in as the output variable, we have that $m=0.85$ and the known input output pair $(s_0,y_0)$ is $(32,35.20)$. Plugging this in to the point-slope form we get:

	\begin{align*}
		y=m(s-s_0)+y_0\\
		y=0.85(s-32)+35.2\\
		y=0.85s-27.2+35.2\\
		y=0.85s+8
	\end{align*}

	Which we can then rewrite in function notation as:
	$$C(s)=0.85s+8$$
}

%%%%%%%%%%%%%%%%%%%%%%%%%%%%%%%%%%%%%%%%%%%%%%%%%%%%%%%%%%%%%
%
% Subsection: Finding Formulas for Linear Functions: Formulas for Linear Models Given Two Input-Output Pairs
%
%%%%%%%%%%%%%%%%%%%%%%%%%%%%%%%%%%%%%%%%%%%%%%%%%%%%%%%%%%%%

\subsection{Formulas for Linear Models Given Two Input-Output Pairs}

Sometimes we are not even lucky enough to have the slope for a linear function. We can, however, still find the formula for a linear function if we have not just one but two input-output pairs for the function.\\

Remember that if we know two input-output pairs for a function, we can use them to calculate its slope. Then, once we have the slope, we can use the methods we’ve already discussed in this chapter to find the formula for the function. If we have the slope and one input-output pair we are set - and in this case, we actually have more than one!

\begin{definition}
	\textbf{\underline{Method for Finding the Formula for a Linear}}\\
	 \textbf\underline{{Function Given Two Input-output Pairs}}
	\begin{enumerate}
		\item Use the two input-output pairs to find the function’s slope
		\item Use the slope together with either one of the input-output pairs to find the function’s formula, using methods previously discussed.
	\end{enumerate}
\end{definition}

Essentially then we are just adding on an additional step at the start - finding the slope. Once we’ve done that, we are pretty much in the same boat as before.

\exam{\label{FormulaLinearFunctionsExample7} Find the formula for the linear function whose graph passes through the points $(2,9)$ and $(-1,24)$.}

\indenttext{We first find the slope:
	\begin{align*}
		m&=\frac{\Delta output}{\Delta input}=\frac{\Delta f(x)}{\Delta x}=\frac{f(2)=f(-1)}{2-(-1)}\\
		\\
		&=\frac{9-24}{2-(-1)}\\
		\\
		&=\frac{-15}{3}\\
		\\
		&=-5
	\end{align*}
	
	We now know the slope and can use it together with either of the two given input-output pairs.  Here we will use $(2,9)$ for no particular reason other than we have to pick one of them. (Using the other pair will work equally well).\\
	
	We’ll use the Back Substitution Method here (though using the point-slope form would work equally well. You’ve got choices!)\\
	
	Since the function is linear we know it can be written in the form:  $$f(x)=mx+b$$
	Plugging in the slope we get:  $$f(x)=-5x+b$$
	And now plugging in the input-output pair and solving for $b$:  
	
	\begin{align*}
		9&=-5(2)+b\\
		9&=-10+b\\
		b=19
	\end{align*}
	
	And finally putting this all together we get: $$f(x)=-5x+19$$
}

Knowing a second input-output pair gives an opportunity to check your answer if you like. If the formula is correct, we know it should give $f(-1)=24$ because $(-1,24)$ is on its graph.  You can verify for yourself that this does indeed work correctly.

\exam{\label{FormulaLinearFunctionsExample8} In patients with diabetes, blood sugar levels often fluctuate dramatically during the day and from day to day. Therefore, measuring a patient’s blood sugar level at any given time does not really reveal very much about overall blood sugar control.\\
	
	A blood test known as \quotes{glycated hemoglobin} or \quotes{A1C} can be performed which measures a patients overall average blood sugar levels over the previous two to three months. It measures the percent of hemoglobin in the blood that has been affected by exposure to glucose. The average blood sugar level is a linear function of the A1C level. An A1C level of $7.0\%$ corresponds to an average blood sugar of $154 mg/dl$ (milligrams of glucose per deciliter of blood). An A1C level of $10.0\%$ corresponds to an average blood sugar level of $241 mg/dl$. (Source: Wikipedia, \quotes{glycated hemoglobin}, accessed 7/24/11)
	
	\begin{enumerate}[label=(\alph*)]
		\item Find a formula for average blood sugar level as a function of A1C.
		\item What would an A1C of 8.2 indicate for an average blood sugar?
		\item A typical average blood sugar for a non-diabetic would be 100. What A1C level would this correspond to?
	\end{enumerate}
}

\indenttext{
	\begin{enumerate}[label=(\alph*)]
		\item Here we know two input-output pairs, and so we can use them to find the slope:
		\begin{align*}
			m&=\frac{\Delta output}{\Delta input}=\frac{\Delta average \text{ } blood \text{ } sugar}{\Delta A1C}\\
			\\
			&=\frac{241-154}{10-7}\\
			\\
			&=\frac{87}{3}\\
			\\
			&=29
		\end{align*}
	
		Now we can use either input-output pair together with this slope to find the vertical intercept and complete the formula. Using $x$ and $f(x)$ for the input variable and function name we get: $$f(x)=mx+b$$
		For no reason other than we’ve got to pick one, we’ll use the pair $(10,241)$:
	
		\begin{align*}
			241&=29(10)+b\\
			241&=290+b\\
			b&=-49
		\end{align*}
	
		Putting this together we get the formula:
		$$f(x)=-29x-49$$
	
		\item Now that we have the formula we can use it. Plugging in $x=8.2$ we get:
	
		\begin{align*}
			f(x)&=29x-49\\
			f(8.2)&=29(8.2)-49\\
			f(8.2)&=188.8
		\end{align*}
	
		So an A1C level of $8.2\%$ indicates an average blood sugar level of $188.8 mg/dl$.
	
		\item Once again using the formula, but this time substituting in an output value:
	
		\begin{align*}
			f(x)&=29x-49\\
			100&=29x-49\\
			151&=29x\\
			x=\frac{151}{29} \approx 5.21.
		\end{align*}
	
		So an average blood sugar of $100 mg/dl$ equates to an A1C level of approximately $5.21\%$. 
	
	\end{enumerate}
}

%%%%%%%%%%%%%%%%%%%%%%%%%%%%%%%%%%%%%%%%%%%%%%%%%%%%%%%%%%%%
%
% Subsection: Formula for Linear Functions: Exercises
%
%%%%%%%%%%%%%%%%%%%%%%%%%%%%%%%%%%%%%%%%%%%%%%%%%%%%%%%%%%%%

\clearpage
\subsection{Exercises}

NOTE: In all of the following problems, when you are asked to find the formula for a linear function, your final answer should be expressed in slope-intercept (that is, $f(x)=mx+b$ form).

\subsubsection*{Formulas for Linear Models Given Slope and Intercept}

\bigskip
\ex{Suppose $f(x)$ is a linear function whose slope is $5$ and whose vertical intercept is $(0, -3)$.  Find a formula for $f(x)$.}
\sol{$f(x)=5x-3$}

\bigskip
\ex{Suppose $f(t)$ is a linear function. Its slope is $-4$ and its vertical intercept is $(0, 9)$. Give a formula for $f(t)$.}

\bigskip
\ex{$g(t)$ is linear with slope $-\frac{3}{5}$ and vertical intercept $\frac{11}{3}$. Give a formula for this function.}
\sol{$g(t)=\frac{3}{5}t+\frac{11}{3}$}

\bigskip
\ex{The linear function $h(z)$ has slope $\frac{1}{4}$ and vertical intercept $-\frac{5}{2}$. Find a formula for this function.}

\bigskip
\ex{A car rental agency charges $\$17.50$ to rent a subcompact car for a day, plus $\$0.20$ per mile.  The cost of a daily rental is a function of the miles driven. Find a formula for this function.}
\sol{$f(x)=0.2x+17.5$}

\bigskip
\ex{To print a textbook, a printer charges a flat $\$5.75$ regardless of its length, plus a charge of $\$0.02$ per page. Give a formula for the charge to print a book as a function of the number of pages it contains.}

\bigskip
\ex{A car is doing $70 mph$ when the driver begins to apply the brakes steadily, decreasing the speed by $0.5$ mph per second. Find a formula for the speed of the car $t$ seconds after the driver
	begins braking.}
\sol{$f(t)=-0.5t+70$}

\bigskip
\ex{I am driving from Rochester NY to Boston MA, a distance of $420 miles$. Assuming that I drive at a steady rate of exactly $60$ miles per hour, the distance I have left to go is a linear function of the time since I started my trip. Find a formula for this function.}

\subsubsection*{Formulas for Linear Models Given Slope and an Input-Output Pair}

\bigskip
\ex{Find a formula for $f(x)$ given that it is a linear function whose slope is $-2$ and its graph passes through the point $(3,5)$.}
\sol{$f(x)=-2x+11$}

\bigskip
\ex{Find a formula for the linear function $g(x)$ which has a slope of $5$ and whose graph passes through $(-3,7)$.}

\bigskip
\ex{Find a formula for the linear function whose slope is $\frac{1}{2}$ and whose graph passes through $(4, -4)$.}
\sol{$f(x)=\frac{1}{2}x-6$}

\bigskip
\ex{Find a formula for a linear function with slope $\frac{-2}{3}$ and whose graph includes the point $(7,1)$.}

\bigskip
\ex{Suppose $h(x)$ is a linear function with slope $-2$ and suppose $h(1) = 3$. Find a formula for this function.}
\sol{$h(x)=-2x+5$}

\bigskip
\ex{Suppose $g(t)$ is a linear function whose slope is $8$. Also suppose $g(3) = -5$. Find a formula for $g(t)$.}

\bigskip
\ex{If $f(\frac{2}{3}) = 5$ and the function is linear with a slope of $21$, what is a formula for $f(t)$?}
\sol{$f(t)=21t-9$}

\bigskip
\ex{If $g(x)$ is linear, its slope is $\frac{-3}{5}$ and $g(-8) = \frac{17}{5}$, what is a formula for $g(x)$?}

\bigskip
\ex{Find a formula for a linear function whose slope is $-\frac{5}{2}$ and which gives an output of $25$ when given an input of $40$.}
\sol{$f(x)=\frac-{5}{2}x+125$}	

\bigskip
\ex{When you input $-7$ into a linear function the resulting output is $12$. The function’s slope is $-\frac{5}{14}$. What is a formula for this function?}

\bigskip
\ex{The appropriate dose (in milligrams) of a medication is a function of the patient’s weight (in pounds). For a $140$ pound patient the dose would be $45$ milligrams. For a heavier or lighter patient, the dose would change at a rate of $0.25$ milligrams per pound. Find a formula for the appropriate dose as a function of patient weight.}
\sol{$f(x)=0.25x+10$}

\bigskip
\ex{For a standard policy, a car insurance company charges a set amount plus an additional $85$ for each point on your license. Suppose you have three points on your license and car insurance costs you $835$. Find a formula for your car insurance cost as a function of points on your license.}

\bigskip
\ex{Water is flowing out of a tank at a constant rate of $25$ liters per hour. After $2$ hours, $365$ liters remain in the tank. Find a formula for the water left in the tank as a function of time. How much water was originally in the tank?}
\sol{$f(x)=-25x+415$; $415$ liters}

\bigskip
\ex{A contractor is sanding hardwood floors to be refinished, working at a steady rate of $420$ square feet per hour. After working for $1 \frac{1}{2}$ hours, she has $1780$ square feet left to do. Find a formula for the remaining flooring to sand as a function of hours worked. How large was the overall job?}

\subsubsection*{Formulas for Linear Models Given Two Input-Output Pairs}

\bigskip
\ex{Find a formula for the linear function whose graph passes through the points $(3, -6)$ and $(5,10)$.}
\sol{$f(x)=8x-30$}

\bigskip
\ex{Find a formula for the linear function whose graph passes through the points $(-20, -400)$ and $(10, 500)$.}

\bigskip
\ex{Find a formula for the linear function whose graph passes through $(50, -300)$ and $(-10, 1500)$.}
\sol{$f(x)=-30x+1200$}

\bigskip
\ex{Find a formula for the linear function whose graph passes through $(-8, 24)$ and $(7, -6)$.}

\bigskip
\ex{$f(x)$ is a linear function and its graph includes the points $(25, 11)$ and $(-50, -19)$. Find a formula for this function.}
\sol{$f(x)=\frac{2}{5}x+1$}

\bigskip
\ex{$f(x)$ is linear and its graph includes $(-20, 18)$ and $(-12, 0)$. Find a formula for $f(x)$.}

\bigskip
\ex{Find a formula for the function whose graph is a straight line passing through the points $(\frac{1}{2}, \frac{5}{6})$ and $(\frac{5}{4}, -\frac{1}{3})$.}
\sol{$f(x)=-{14}{9}x+\frac{29}{18}$}

\bigskip
\ex{The graph of $g(t)$ is a straight line and its graph includes $(3, -\frac{1}{2})$ and $(\frac{5}{4}, 3)$. Find a formula for $g(t)$.}

\bigskip
\ex{The cost of shipping a load of freight includes a set base fee charged regardless of the amount shipped plus a set cost per ton. If it costs $\$800$ to ship $2.5$ tons and it costs $\$2000$ to ship $10$ tons, 
	\begin{enumerate}[label=(\alph*)]
		\item find a formula for shipping cost as a function of tons shipped, 
		\item determine the cost to ship $50$ tons, and 
		\item determine how much a shipment weighed if it cost $\$3000$ to ship.
	\end{enumerate}
}
\sol{a.  $f(x)=160x+400$\\
	b.  $\$8400$\\
	c.  $16.25$ tons}

\bigskip
\ex{A candy vending machine dispenses a set weight of jellybeans each time someone puts a quarter in. Last Tuesday the machine contained $84$ ounces of jellybeans and seven quarters.  One week later, the machine contained $63$ ounces of jellybeans and forty-nine quarters. No one refilled the machine or removed any money from it in between. 
	\begin{enumerate}[label=(\alph*)]
		\item Find a formula for the weight of jellybeans in the machine as a function of money collected. 
		\item How much candy do you get for each quarter? 
		\item How much candy does the machine contain when it is full?
	\end{enumerate}
}

\bigskip
\ex{A manufacturing line takes some time to set up, but once this is done it produces widgets at a steady rate. Including start-up time, it takes $4$ hours to produce $500$ crates of widgets, and $7$ hours to produce $1100$ crates.
	\begin{enumerate}[label=(\alph*)]
		\item Find a formula for crates of widgets produced as a function of time. 
		\item How many widgets could be produced in 12 hours? 
		\item How long would it take to produce 1500 crates?
	\end{enumerate} 
}
\sol{a.  $f(x)=200x-300$\\
	b.  $2100$ crates\\
	c.  $9$ hours}

\bigskip
\ex{The total amount of insulin a diabetic patient requires in a day is a function of the amount of carbohydrates he consumes. For a given patient, the body requires a certain amount of insulin regardless of the amount consumed, and then there is a set amount of insulin (in units) required per gram of carbohydrate. If a particular diabetic patient consumes $300$ grams of carbohydrates, he will need $74$ units of insulin. He’ll need $99$ units if he eats $450$ grams of carbohydrate. 
	\begin{enumerate}[label=(\alph*)]
		\item Find a formula for insulin required as a function of carbohydrates consumed. 
		\item Explain the meaning of the vertical intercept of this function. 
		\item How much insulin would be required to cover a daily consumption of $250$ grams of carbohydrate?
	\end{enumerate}
}

\subsection*{Grab Bag}

\bigskip
\ex{The boiling point of water is $212^\circ$F at sea level. At $1000$ feet above sea level the boiling point is approximately $210^\circ$F. Assuming that boiling point is a linear function of altitude above sea level (in reality it isn’t, but it is close enough to linear that we can get a very good approximation by assuming it is anyway) find a formula for the boiling point of water as a function of altitude above sea level.}
\sol{$f(x)=-0.002x+212$}

\bigskip
\ex{$f(x)$ is linear, $f(3) = 70$ and $f(6) = -50$. 
	\begin{enumerate}[label=(\alph*)]
		\item Find a formula for $f(x)$. 
		\item Find $f(4)$.
	\end{enumerate}
}

\bigskip
\ex{If $g(x)$ is linear, $g(0) = 2$ and $g(3) = 11$, find a formula for $g(x)$.}
\sol{$g(x)=3x+2$}

\bigskip
\ex{Laundry detergent is sold in a pump container. Each load of laundry requires $4$ presses of the pump. When full, the container contains enough detergent for $200$ pump presses.
	\begin{enumerate}
		\item Find a formula for the number of pump presses remaining in the container as a function of the number of loads of laundry done. 
		\item Find and interpret the horizontal intercept of this function.
	\end{enumerate}
}

\bigskip
\ex{This question is based on a poster at an amusement park advertising $\$0.99$ refills after buying a souvenir cup for $\$6.99$ with the initial fill-up.
	\begin{enumerate}[label=(\alph*)]
		\item Find a formula for the total cost of your drinks with this offer as a function of the number of times you refill the cup.
		\item Find a formula for the total cost of your drinks with this offer as a function of the number of times the cup is filled.
		\item For (a) and (b) you should have different values for the vertical intercept. What does the vertical intercept mean in each of these cases? What is the difference between them?
	\end{enumerate}
}
\sol{a.  $f(x)=0.99x+6.99$\\
	b.  $f(x)=0.99x+6$\\
	c.  In (a) the vertical intercept was the initial cost of the cup including the first fill-up ($x$ is the number of refills); In (b) the vertical intercept was the cost of the cup not including the first fill ($x$ is the number of times the cup is filled, including the initial fill-up). }

\bigskip
\ex{Find a formula for the function whose graph is the straight line passing through $(-\frac{1}{2}, 3)$ and $(\frac{3}{2}, -5)$.}

\bigskip
\ex{Airaloft Airlines offers a credit card, where users receive a set number of frequent flier miles for each dollar they charge to the card. If you charge a total of $\$20,000$ you would earn $40,000$ miles. If you charge a total of $\$10,000$ you would earn $25,000$ miles.
	\begin{enumerate}[label=(\alph*)]
		\item Find a formula for the miles earned as a function of dollars charged. 
		\item How many miles would you earn if you charge $\$30,000$? 
		\item A free trip to Australia requires $80,000$ miles. How much would you need to charge to your card to earn enough for that trip? 
		\item Give a possible interpretation for the vertical intercept of this function.
	\end{enumerate} 
}
\sol{a.  $f(x)=1.5x+10000$\\
	b.  $55,000$ miles\\
	c.  $\$46,666.67$\\
	d.  You earn $10,000$ miles for opening a charge card account.}

\bigskip
\ex{Property taxes in the Town of Townsville are calculated by charging a flat amount for each property, plus a set rate per square foot of living space. A $1200$ square foot house would be charged $\$2400$ in taxes. A $2000$ square foot house would pay $\$3800$. What would the taxes be for an $1800$ square foot house?
}

\bigskip
\ex{$h(t)$ is linear, $h(5) = \frac{7}{5}$ and $h(-2) = -\frac{14}{5}$. Find a formula for $h(t)$.}
\sol{$h(t)=\frac{3}{5}t-\frac{8}{5}$}

\bigskip
\ex{The upward velocity of a model rocket is a linear function of the time since it was launched.  The rocket’s initial upward velocity is $192$ feet per second. After $2$ seconds the upward velocity has dropped to $128$ feet per second. Find a formula for upward velocity as a function of time.
}

\bigskip
\ex{Find a formula for a linear function whose slope is $9$ and whose vertical intercept is located at $(0, -8)$.}
\sol{$f(x)=9x-8$}

\bigskip
\ex{Find a formula for the linear function with slope $-2$ and whose vertical intercept is the origin.}

\bigskip
\ex{Suppose $f(x)$ is a linear function, some of whose input-output pairs are given in the table below:
	\begin{center}
		\begin{tabular}{c|c}
			x & f(x) \\
			\hline
			3.5 & 12 \\
			5 & 15 \\
			8.5 & 22
		\end{tabular}
	\end{center}
	Find a formula for this function.
}
\sol{$f(x)=2x+5$}

\bigskip
\ex{A fruit wholesaler charges grocers $\$0.40$ per pound for bananas, plus a delivery charge which is the same set amount regardless of how many bananas he delivers. Yummi’s Greenmarket bought $350$ pounds of bananas and paid $\$165$. What is the delivery charge? Find a formula for the cost of an order of $x$ pounds of bananas.
}

\bigskip
\ex{An internet discount coupon promoter charges merchants a flat $\$50$ fee to send out an email promotion, and then receives a set amount for each discount coupon purchased. If $40$ were purchased, Harry’s Haberdashery would pay a total of $\$250$. Find a formula for the total Harry’s would pay as a function of discount coupons purchased, and use this formula to determine how much they would pay if $84$ were purchased.
}
\sol{$C(x)=50+5x$}

\bigskip
\ex{Linear functions by definition must have a constant rate of change. What this means in a given application, though, can lead to some confusion. For example, suppose that a small ice cream company is trying to mathematically model its profits. The company has fixed costs (that is, costs that they have to pay regardless of how much ice cream they sell) of $\$4,000$ per month. They sell their ice cream for $\$3.50$ per pint, and it costs them $\$2.25$ to make each pint of ice cream.
	Ben says: \quotes{Well, our profit is clearly a linear function of the number of pints of ice cream we sell. Since $\$3.50$ - $\$2.25$ = $\$1.25$, we make the same, steady $\$1.25$ per pint sold. Since our profits as a function of pints sold increase at a constant rate, the function is linear.}\\
	Jerry says: \quotes{Nonsense! I agree that if our profit per pint sold were constant, the function would be linear, but it’s not! If we sell, say, $5000$ pints, we take in $5000(\$3.50) = \$17,500$. It costs us $5000(\$2.25) = \$11,250$ to make that ice cream, plus our fixed costs are $\$4,000$, for a total of $\$15,250$. So we make $\$17,500 - \$15,250 = \$2,250$ profit. And that would be $\$2,250/5000$ pints = $\$0.45$ per pint. But now if we sell just $2000$ pints, we take in $2000(\$3.50) = \$7,000$. That ice cream costs $2000(\$2.25) = \$4,500$ to make, and so when you add in our fixed costs our costs are $\$8,500$ in total. So our \quotes{make} $\$7,000 - \$8,500 = -\$1,500$. So our \quotes{profit} is  $-\$1,500/2000 = -\$0.75$ per pint. A loss of $75$ cents a pint sure ain’t the same as a profit of $\$0.45$ per pint!}\\
	Who is right? Explain.
}

\clearpage

